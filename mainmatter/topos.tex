\chapter{Epilogue: towards realizability toposes}\label{chap:topos}

In this epilogue we briefly discuss what is perhaps the elephant in the room (pun
intended\footnote{For the uninitiated: Johnstone's two volume compendium on
  topos theory is titled \citetitle{Elephant} and commonly referred to as ``the
  Elephant''. A third volume, covering realizability toposes was planned but has
  yet to be published.}): the \textbf{realizability topos} associated to a pca.
%
Indeed, the notion of a realizability topos is noticeably absent from these
notes, essentially because its construction is relatively complicated compared
to the category of assemblies.

Working with an (elementary) topos has some benefits of course. The primary
feature of a topos is the existence of a classifier
\(\top \colon 1 \to \Omega\) for \emph{all} monomorphisms---the category of
assemblies only has classifiers for special subclasses (e.g.\
\(\lnot\lnot\)-stable or decidable monos) as we have seen.
%
A consequence is that every monomorphism is regular (because any mono into \(X\)
is the equalizer of its characteristic map \(\chi \colon X \to \Omega\) and the
composite \(X \to \One \xrightarrow{\top} \Omega\)).
%
Thus, while regular monos and monos are distinct in the category of assemblies,
the passage to the realizability topos unifies these two concepts.

Here we briefly sketch the construction of the realizability topos over a pca.
The objects of the realizability topos are \textbf{partial equivalence
  relations}: sets \(X\) equipped with a \(\powerset(\AA)\)-valued predicate
\({\sim} : X \times X \to \powerset(\AA)\) such that transitivity and symmetry
of \(\sim\) are ``realized''.
%
For the latter we set up a Heyting prealgebra of \(\powerset(\AA)\)-valued
predicates similar to that of the realizability predicates
in~\cref{sec:cat-logic-in-asm} with the notable exception that we now work with
bare sets, so the elements of our sets are not equipped with realizers.
%
In fact, while the relation \(\sim\) is not assumed to be reflexive in
general, elements of \(x \sim x\) are taken as realizers of \(x\) and
evidence that \(x\) ``exists''.
%
In general, we think of \({\sim}\) as an equality predicate on \(X\).

A morphism between such pairs \((X,\sim_X)\) and \((Y,\sim_Y)\) is defined
to be a \(\powerset(\AA)\)\nobreakdash-valued predicate \(F\) on \(X \times Y\)
such that the statement that \(F\) is a \emph{strict} functional relation are
realized.  Strictness means
\(\forall(x\in X).\forall(y\in Y).\pa*{F(x,y) \Rightarrow x \sim_X x \land y
  \sim_Y y}\), that is, \(F\) relates only elements that ``exist''.
%
Moreover, \(F\) should respect the equality predicates, i.e.\
\(\forall(x,x'\in X).\forall(y,y' \in Y).\pa*{F(x,y)\land(x\sim_X
  x')\land(y\sim_Y y') \Rightarrow F(x',y')}\) must be realized.
%
Finally, \(F\) should be a \emph{functional} relation, so we also require
single-valuedness and totality, in the sense that
\(\forall(x\in X).\forall(y,y' \in Y).\pa*{F(x,y) \land F(x,y') \to y \sim_Y
  y'}\) and \(\forall(x \in X).(x \sim_X x \to \exists(y \in Y).F(x,y))\) are
realized.

One can show that the resulting category is indeed a topos and that it houses
the category of assemblies as a full subcategory. In fact, the category of
assemblies is precisely the subcategory of \(\lnot\lnot\)-separated objects of
the realizability topos, i.e.\ those objects \(X\) for which
\(\forall(x,y:X).(\lnot\lnot(x=y) \to (x=y))\) holds in the internal logic.
%
This is perhaps foreseen by the ``classical'' interpretation of equality in the
realizability logic of assemblies (\cref{realizability-logic}).

The realizability topos over Kleene's first model is known as \textbf{the
  effective topos} as was detailed in Hyland's landmark paper~\cite{Hyland1982}.
%
The more general theory, known as \textbf{tripos theory}, was the subject of
Pitts's PhD thesis~\cite{Pitts1981} and his joint paper with Hyland and
Johnstone~\cite{HJP1980}.
%
The interested reader can consult van Oosten's comprehensive
textbook~\cite{vanOosten2008} and Zoethout's master's
thesis~\cite{Zoethout2018}.

%%% Local Variables:
%%% mode: latexmk
%%% TeX-master: "../main"
%%% End:
