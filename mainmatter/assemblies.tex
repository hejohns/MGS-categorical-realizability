\chapter{Categories of assemblies}\label{chap:assemblies}

In this chapter we describe the \emph{category of assemblies} over a fixed but
arbitrary pca.
%
In particular, we show that the category has many desirable properties, for
example it is (locally) cartesian closed and has finite colimits.
%
In fact, the category has sufficient structure to support an interpretation of
first order logic as we explore in~\cref{chap:logic}.

Roughly speaking, the category of assemblies has sets with computability
data as objects and computable functions between such sets as morphisms.
%, where the
%computability is with respect to the chosen pca.
%
The category of assemblies as a whole may be thought of as a world of
computable mathematics.

\begin{definition}[Assembly]
  An \textbf{assembly} over a pca \(\AA\) is a set \(X\) together with a
  relation \({\realizes}\) between \(\AA\) and \(X\) such that for all
  \(x \in X\), there exists at least one element \(\pca{a} \in \AA\) with
  \(\pca{a} \realizes x\).
\end{definition}

The relation \(\pca{a} \realizes x\) is pronounced as ``\(\pca{a}\)
\textbf{realizes} \(x\)'' and we also say that \(\pca{a}\) is a
\textbf{realizer} of \(x\). We think of \(\pca{a}\) as an \emph{implementation}
of \(x \in X\) in the pca.
%
The requirement on assemblies is that each element of the set must have at least
one implementation.

\begin{notation}[\(\carrier{X}\), \({\realizes_X}\)]
  Given an assembly \(X\), we will write \(\carrier{X}\) for its underlying set
  and \(\realizes_X\) for its relation between \(\AA\) and \(\carrier{X}\).
\end{notation}

\begin{example}[Assembly of booleans, \(\Two\)]\label{ex:assembly-of-booleans}
  The \textbf{assembly of booleans}, denoted by \(\Two\), is defined as
  \[
    \carrier{\Two} \coloneqq \set{0,1}
    \quad\text{with realizers}\quad
    \pcafalse \realizes_\Two 0
    %\quad
    \text{ and }%\quad
    \pcatrue \realizes_\Two 1,
  \]
  where we recall the booleans \(\pcafalse\) and \(\pcatrue\) from \cref{sec:booleans}.
\end{example}

\begin{example}[Assembly of natural numbers, \(\NatAsm\)]\label{ex:NatAsm}
  The \textbf{assembly of natural numbers}, denoted by \(\NatAsm\), is defined as
  \[
    \carrier{\NatAsm} \coloneqq \Nat
    \quad\quad\text{and}\quad\quad
    \numeral{n} \realizes_\NatAsm n \text{ for each \(n \in \Nat\)},
  \]
  where we recall from~\cref{sec:numerals} that \(\numeral{n}\) is the
  \(n\)\textsuperscript{th} Curry numeral.
\end{example}

\begin{example}
  Taking Kleene's first model as our pca, \(\AA = \Kone\), we can consider the
  assembly \(X\) of Turing computable functions:
  \[
    \carrier{X} \coloneqq \set{f \colon \Nat \to \Nat \mid f \text{ is Turing computable}}
    \quad\quad\text{and}\quad\quad
    m \realizes_X f \iff \prenum{m} = f
  \]
  where we recall \(\Kone\) and \(\prenum{-}\) from~\cref{ex:Kleene-1}.
  %
  We remark that each \(f \in \carrier{X}\) has infinitely many realizers.

  Notice that, with this realizability relation, we cannot let \(\carrier{X}\)
  be the set of \emph{all} functions from \(\Nat\) to \(\Nat\), because then the
  set of realizers of a noncomputable function would be empty, which is not
  allowed by the definition of an assembly.
\end{example}

In the literature, assemblies over \(\Kone\) are sometimes referred
to as \emph{\(\omega\)-sets}.

\section{Morphisms of assemblies}

As mentioned in the opening paragraphs of this chapter, we wish to organize the
assemblies into a category whose morphisms are ``computable'' functions between
the underlying sets of assemblies.
%
The computability requirement is made precise by requiring the existence of a
\emph{tracker} which we define now.

\begin{definition}[Track]
  For assemblies \(X\) and \(Y\), we say that an element \(\pca{t} \in \AA\)
  \textbf{tracks} a function \(f \colon \carrier{X} \to \carrier{Y}\) if for all
  \(x \in \carrier{X}\) and \(\pca{a} \in \AA\), if \(\pca{a} \realizes_X x\), then
  \(\pca{t}\pca{a}\) is defined and \(\pca{t}\pca{a} \realizes_Y f(x)\).
\end{definition}

\begin{notation}
  We will shorten the above to: ``\(\pca{t}\pca{a} \realizes_Y f(x)\)
  for all \(x \in \carrier{X}\) and \(\pca{a} \realizes_X x\)''.
  %
  That is, we implicitly quantify over \(\pca{a}\) and we implicitly assume that
  \(\pca{t}\pca{a}\) is defined when we write
  \(\pca{t}\pca{a} \realizes_Y f(x)\).
\end{notation}

\begin{definition}[Assembly map]
  An \textbf{assembly map} from an assembly \(X\) to an assembly \(Y\) is a
  function \(f \colon \carrier{X} \to \carrier{Y}\) that is tracked by some
  element.
\end{definition}

The existence of a tracker is a required \emph{property} of an assembly map and
\emph{not} part of the data, so an assembly map is (unlike an assembly)
\emph{not} a pair of a function and a tracker; it is just a function for which
there exists some tracker.

\begin{proposition}
  Assemblies and assembly maps form a category with composition given
  by composition of functions on underlying sets.
\end{proposition}
\begin{proof}
  We need to verify that composition is well defined, i.e., that if
  \(f \colon X \to Y\) and \(g \colon Y \to Z\) are assembly maps, then
  \(g \circ f \colon \carrier{X} \to \carrier{Z}\) is tracked.
  %
  Let \(\pca{t_f}\) and \(\pca{t_g}\) track \(f\)~and~\(g\), respectively.  We
  claim that \(\lambdapca{x}{\pca{t_g}(\pca{t_f}(x))}\) tracks \(g \circ
  f\). Indeed, the closed term \(\lambdapca{x}{\pca{t_g}(\pca{t_f}(x))}\) is
  defined by construction, and if \(\pca{a} \realizes_X x\), then
  \[
    {(\lambdapca{x}{\pca{t_g}(\pca{t_f}(x)}))\pca{a} =
    \pca{t_g}(\pca{t_f}\pca{a})} \realizes_Z g(f(x))
  \]
  by choice of \(\pca{t_f}\) and \(\pca{t_g}\).
  %
  Moreover, for each assembly \(X\), we have an identity morphism on \(X\) given
  by the identity on \(\carrier{X}\) and tracked by \(\icomb\).
  %
  Finally, associativity of composition holds because composing functions of
  sets is associative.
\end{proof}

\begin{notation}[\(\Asm{\AA}\)]
  We write \(\Asm{\AA}\) for the category of assemblies over a pca \(\AA\).
\end{notation}

\begin{example}
  For the trivial pca \(\AA = \set{\singleton}\) we recover the familiar
  category \(\Set\) of sets and functions as \(\Asm{\AA}\).
\end{example}

\begin{remark}[Relative categories of assemblies]
  An interesting variation on the category of assemblies is obtained if we start
  with a pca \(\AA\) and an elementary sub-pca \(\AA^\#\)
  (recall~\cref{def:elementary-sub-pca} and \cref{Scott-re,Kleene-rec-2}).
  %
  While we still ask that the realizers of elements of sets come from the
  larger pca \(\AA\), we now require the trackers of assembly maps to come from
  the smaller sub-pca \(\AA^\#\) instead.
  %
  The requirements on an elementary sub-pca guarantee that this is again a
  category which we call the \textbf{relative category of assemblies}.
  %and denote
  %by \(\Asm{\AA,\AA^\#}\).

  Especially in the typical examples (\ref{Scott-re} and \ref{Kleene-rec-2})
  where \(\AA = \Ktwo\) and \(\AA^\# = \Ktworec\), or \(\AA = \Scott\) and
  \(\AA^\# = \Scottre\), the idea of the relative categories of assemblies is
  nicely captured by Bauer's~\cite[p.~36 and 45]{Bauer2006,Bauer2023}
  slogan
  \begin{center}
    \emph{Topological data --- computable functions!}
  \end{center}
  %nicely captures the intent of the relative categories of assemblies.
\end{remark}

\section{Categorical constructions}

This section shows that the category of assemblies has a rich categorical
structure. In particular, we will construct finite (co)products, exponentials
(in slices) and (co)equalizers of assemblies.
%
This structure is important for the interpretation of first order logic in
assemblies that we explore further in \cref{chap:logic}.
%
But besides this, presenting the required constructions provides excellent
opportunities for improving our understanding of the category of assemblies.

\subsection{Cartesian closure and equalizers}
We start by showing that the category of assemblies is cartesian closed and has
equalizers. The description of the latter will prove useful in our study of
regular monomorphisms of assemblies (\cref{sec:regular-monos}).

\begin{proposition}[Terminal object]
  The terminal object \(\One\) in \(\Asm{\AA}\) is given by
  \[
    \carrier{\One} \coloneq \set{\singleton}
    \quad
    \text{and}
    \quad
    \pca{a} \realizes_\One \singleton
    \text{ for all } \pca{a} \in \AA.
  \]
\end{proposition}
\begin{proof}
  As in \(\Set\).
\end{proof}

The pairing and projection combinators from~\cref{sec:basic-programming-in-pcas}
are essential in the construction of finite products of assemblies.

\begin{proposition}[Products]
  The product \(X \times Y\) of two assemblies \(X\)~and~\(Y\) is given by
  \[
    \carrier{X \times Y} \coloneq \carrier{X} \times \carrier{Y}
    \quad\text{and}\quad
    \pcapair\pca{a}\pca{b} \realizes_{X \times Y} (x,y)
    \text{ for }
    \pca{a} \realizes_X x
    \text{ and }
    \pca{b} \realizes_Y y.
  \]
\end{proposition}
\begin{proof}
  The projection maps \(\pi_1 \colon X \times Y \to X\) and
  \(\pi_2 \colon X \times Y \to Y\) are given by \((x,y) \mapsto x\) and
  \((x,y) \mapsto y\), and tracked by \(\pcafst\) and \(\pcasnd\), respectively.
  %
  Moreover, every pair of assembly maps \(f \colon Z \to X\) and
  \(g \colon Z \to Y\) induces an assembly map
  \(\langle f,g\rangle \colon Z \to X \times Y\) given by
  \(z \mapsto (f(z),g(z))\) and tracked by
  \(\lambdapca{u}{\pcapair(\pca{t_f}u)(\pca{t_g}u)}\) when \(\pca{t_f}\) and
  \(\pca{t_g}\) track \(f\) and \(g\), respectively.
\end{proof}

So far, the underlying sets of the terminal object and product of two assemblies
have been exactly as in \(\Set\), e.g. \(\carrier{X \times Y}\) is the product
of the two sets \(\carrier{X}\) and \(\carrier{Y}\) in \(\Set\).
%
A notable exception to this is the exponential (a.k.a.\ ``internal hom''):
the object of morphisms between two assemblies.
%
It would not make sense for the carrier of the exponential of assemblies \(X\)
and \(Y\) to consist of all functions from \(\carrier{X}\) to \(\carrier{Y}\),
because (a) the exponential is usually similar to the hom-set of morphisms from
\(X\) to \(Y\), and (b) it would not be clear what the realizers of an arbitrary
function between carriers should be.
%
Instead, the exponential is given by assembly maps only.

\begin{proposition}[Exponentials]
  The exponential \(Y^X\) of two assemblies \(X\)~and~\(Y\) is given by
  \[
    \carrier*{Y^X} \coloneq
    \text{the set of assembly maps from \(X\) to \(Y\)}
    \quad\text{and}\quad
    \pca{t} \realizes_{Y^X} f
    \text{ if \(\pca{t}\) tracks \(f\)}.
  \]
\end{proposition}
\begin{proof}
  The evaluation morphism \(\operatorname{ev} \colon Y^X \times X \to Y\) given
  by \((f,x) \mapsto f(x)\) is tracked by
  \(\lambdapca{u}{\pcafst u(\pcasnd u)}\).
  %
  Moreover, every \(g \colon Z \times X \to Y\) induces a unique assembly map
  \(\tilde g \colon Z \to Y^X\) making the diagram
  \[
    \begin{tikzcd}
      Y^X \times X \ar[rr,"\operatorname{ev}"]
      & & Y \\
      & Z \times X \ar[ur,"g"']
      \ar[ul,"{\tilde g} \,\times\, {\id_X}"]
    \end{tikzcd}
  \]
  commute.
  %
  Indeed, there is a unique assignment
  \(\tilde g(z) \coloneq (x \mapsto g(z,x))\) and this assignment is tracked by
  \(\lambdapca{u}{(\lambdapca{v}{\pca{t_g}(\pcapair u\,v)})}\) when
  \(\pca{t_g}\) tracks \(g\).
\end{proof}

Thus, we conclude that the category \(\Asm{\AA}\) is cartesian closed.

Finally, we construct equalizers in \(\Asm{\AA}\). Their description will come
in useful when we study regular monomorphisms in \cref{sec:regular-monos}. We
also note that we obtain an explicit construction of pullbacks when combined
with the construction of products.

\begin{proposition}[Equalizers]
  The equalizer \(E\) of two assembly maps \(f,g \colon X \to Y\) is given by
  \[
    \carrier{E} \coloneq \set{x \in \carrier{X} \mid f(x) = g(x)}
    \quad\text{and}\quad
    \pca{a} \realizes_E x \text{ if } \pca{a} \realizes_X x.
  \]
\end{proposition}
\begin{proof}
  On the level of underlying sets, this is as in the category of sets, so it
  suffices to show that the relevant functions are tracked.
  %
  The inclusion \(i \colon \carrier{E} \to \carrier{X}\) is tracked by
  \(\icomb\).
  %
  Given an assembly map \(h \colon D \to X\) such that
  \(f \circ h = f \circ h\), the map \(h \colon \carrier{D} \to \carrier{X}\)
  factors uniquely through \(i\) via \(k \colon \carrier{D} \to \carrier{E}\)
  which is tracked by any tracker of \(h\).
\end{proof}

\subsection{Coproducts and coequalizers}
%
We now move on to colimits in the category of assemblies.

\begin{proposition}[Initial object]
  The initial object \(\Zero\) in \(\Asm{\AA}\) is given by
  \( \carrier{\Zero} \coloneq \emptyset \) with the empty realizability
  relation.
\end{proposition}
\begin{proof}
  As in \(\Set\).
\end{proof}

\begin{proposition}[Coproducts]\label{coproducts}
  The coproduct \(X + Y\) of two assemblies \(X\) and \(Y\) is given by
  \begin{align*}
    \carrier{X + Y} \coloneq \carrier{X} + \carrier{Y}
    \quad\text{and}\quad
    \pcaleft\pca{a} &\realizes_{X + Y} \inl(x)
      \text{ for } \pca{a} \realizes_X x,
    \\
    \pcaright\pca{b} &\realizes_{X + Y} \inr(y)
    \text{ for }
    \pca{b} \realizes_Y y,
  \end{align*}
  where
  \[
    \pcaleft \coloneqq \pcapair\pcafalse
    \text{ and }
    \pcaright \coloneqq \pcapair\pcatrue.
  \]
\end{proposition}

Notice that the disjointness of the coproduct is witnessed by tagging the
realizers with booleans.

\begin{exercise}\label{exer:coproducts}
  Prove~\cref{coproducts}.

  \emph{Warning}: Carefully check that the closed terms you give for the
  trackers are defined and thus give elements of the pca as required.
\end{exercise}

While there are several interesting assemblies whose carrier is the set
\(\set{0,1}\) (as we will see in \cref{chap:logic}), the following exercise
justifies the notation \(\Two\) for the assembly of booleans as defined in
\cref{ex:assembly-of-booleans}.

\begin{exercise}\label{exer:coproduct-booleans}
  Show that \({\One + \One} \cong \Two\).
\end{exercise}

Finally, we construct coequalizers in \(\Asm{\AA}\). Their description will be
useful in our study of regular epimorphisms (\cref{sec:regular-epis}).

\begin{proposition}[Coequalizers]
  The coequalizer \(C\) of assembly maps \(f,g \colon X \to Y\) is given by
  \[
    \carrier{C} \coloneq \carrier{Y}/{\sim}
    \quad\text{and}\quad
    \pca{a} \realizes_C [y] \text{ if } \pca{a} \realizes_Y y'
    \text{ for some } {y'} \sim y,
  \]
  where \({\sim}\) is the least equivalence relation on \(\carrier{Y}\)
  generated by \(f(x) \sim g(x)\) for all \(x \in \carrier{X}\).
  % setting \(y \sim y'\) whenever there exists \(x \in \carrier{X}\)
  % with \(f(x) = y\) and \(g(x) = y'\).
\end{proposition}
\begin{proof}
  On the level of underlying sets, this is as in the category of sets, so it
  suffices to show that the relevant functions are tracked.
  %
  The quotient map \(q \colon \carrier{Y} \to \carrier{Y}/{\sim}\) is tracked by
  \(\icomb\).
  %
  Given an assembly map \(h \colon Y \to Z\) such that
  \(h \circ f = h \circ g\), the map \(h \colon \carrier{Y} \to \carrier{Z}\)
  factors uniquely through \(q\) via
  \(k \colon \carrier{Y}/{\sim} \to \carrier{Z}\) with \(k([y]) \coloneq h(y)\).
  %
  Moreover, the function \(k\) is tracked by any tracker of \(h\).
\end{proof}

\subsection{Natural numbers object}
The notion of natural numbers can be captured categorically via a universal
property which is due to Lawvere~\cite{Lawvere1963}. In a category
\(\mathcal C\), a \textbf{natural numbers object (nno)} is an object \(N\)
equipped with morphisms \(z \colon 1 \to N\) (``zero'') and \(s \colon N \to N\)
(``successor'') such that for all triples
\((X,x \colon 1 \to X,f\colon X \to X)\) there is a unique morphism
\(r \colon N \to X\) (defined by ``recursion'') making the diagram
\[
  \begin{tikzcd}[row sep=12mm,column sep=12mm]
    & N \ar[r,"s"] \ar[d,dashed,"r"] & N \ar[d,dashed,"r"] \\
    1 \ar[ur,"z"] \ar[r,"x"] & X \ar[r,"f"] & X
  \end{tikzcd}
\]
commute.

\begin{exercise}\label{exer:nno} \leavevmode
  \begin{enumerate}[(i)]
  \item Exhibit \(\Nat\) as a nno in \(\Set\).
  \item Exhibit \(\NatAsm\) (from~\cref{ex:NatAsm}) as a nno in \(\Asm{\AA}\).

    \emph{Hint}: Use~\cref{exer:primitive-recursion}.
  \end{enumerate}
\end{exercise}

Looking ahead to \cref{chap:logic} we note that if we wish to interpret
arithmetic in the category of assemblies, then its natural numbers object serves
as the interpretation of the sort of natural numbers.

\subsection{Dependent products and sums}

For the purposes of these lecture notes, a proof sketch of the following result
suffices. We discuss the significance of the result below.

\begin{proposition}\label{base-change-adjoints}
  For every morphism \(f \colon X \to Y\) of assemblies, the pullback functor
  \(f^\ast \colon \Asm{\AA}/Y \to \Asm{\AA}/X\) has both a left adjoint
  \(\sum_f\) and a right adjoint \(\prod_f\).
\end{proposition}
\begin{proof}[Proof sketch]
  We only describe the constructions and leave the verification of the details
  to the interested reader.
  %
  The left adjoint \(\sum_f\) takes an object \(g \colon Z \to X\) of
  \(\Asm{\AA}/X\) to the object \(f \circ g\) of \(\Asm{\AA}/Y\).
  %
  On morphisms it is the identity.

  The right adjoint \(\prod_f\) is more involved.
  %
  Given an object \(g \colon Z \to X\) of \(\Asm{\AA}/X\), we consider the
  assembly \(P\) of ``fiberwise maps''. It is given by
  \[
    \carrier{P} \coloneq
    \set{(y,s) \mid s \colon f^{-1}(y) \to Z \text{ such that }
      \forall(x \in \carrier{f^{-1}(y)})\,.\,s(x) \in \carrier{g^{-1}(x)}},
  \]
  where
  \[
    \carrier{f^{-1}(y)} \coloneq \set{x \in \carrier{X} \mid f(x) = y}
    \quad\text{with realizers}\quad
    \pca{a} \realizes_{f^{-1}(y)} x \text{ if } \pca{a} \realizes_X x,
  \]
  (and similarly for \(g^{-1}(x)\)), and for realizers, we put
  \[
    \pcapair\pca{b}\pca{t} \realizes_P (y,s)
    \text{ if }
    \pca{b} \realizes_Y y
    \text{ and}
    \pca{t} \text{tracks } s.
  \]
  Now, \(P\) defines an object of \(\Asm{\AA}/Y\) by considering the first
  projection \(\pi_1 \colon P \to Y\) which is tracked by \(\pcafst\).
  %
  This projection map is the value of \(\prod_f(g)\).
  %
  Given a morphism
  \[
    \begin{tikzcd}[row sep=4mm,column sep=4mm]
      Z \ar[rr,"k"] \ar[dr,"g"'] & & W \ar[dl,"h"] \\
      & X
    \end{tikzcd}
  \]
  in \(\Asm{\AA}/X\), we define \(\prod_f(k) \colon \prod_f(g) \to \prod_f(h)\)
  as a function on sets by \((y,s) \mapsto (y, h \circ s)\). This assignment can
  be shown to be tracked because \(h\) and \(s\) are.
\end{proof}

The above proposition is equivalent to the fact that the category \(\Asm{\AA}\)
is \emph{locally cartesian closed}, i.e.\ that each slice category
\(\Asm{\AA}/X\) is cartesian closed.
%
One may check that the exponential \(g^f\) of two objects
\(f,g \in \Asm{\AA}/X\) is given by \(\prod_f(f^\ast(g))\).

The functors \(\sum_f\) and \(\prod_f\) also satisfy the so-called
Beck--Chevalley condition~\cite[Theorem~4.4]{Streicher2018}, which we don't
spell out here. Intuitively, this condition expresses that
\(\prod\)~and~\(\sum\) preserve substitution which is given by pullback (see
e.g.~\cite{Bauer2012} for an informal explanation of the latter).

The adjoints allow us to interpret Martin-L\"of dependent type theory, where, as
the notation suggests, dependent sums and products are interpreted using
\(\sum\) and \(\prod\), respectively. The interested reader may
consult~\cite{Seely1984,Jacobs1999,Streicher1991} to learn more.
%
We implicitly rely on (a slight variation of) \cref{base-change-adjoints} for
the interpretation of first order logic in the category of assemblies in
\cref{chap:logic}, where we only need the adjoints for \(f\) a projection
map.
%
But our presentation in \cref{chap:logic} will spell things out in more concrete
terms, so understanding \cref{base-change-adjoints} in detail is not strictly
required.

\section{Relation to the category of sets}\label{sec:relation-to-Set}

We introduce an adjunction
\[
\begin{tikzcd}
\mathbb{\Asm{\AA}}
\arrow[r, "\Gamma"{name=F}, shift left=2mm] &
\mathbb{\Set}
\arrow[l, "\nabla"{name=G}, shift left=2mm]
\arrow[phantom, from=F, to=G, "\dashv" rotate=-90]
\end{tikzcd}
\]
relating the categories of sets and assemblies.

\begin{definition}[Forgetful functor, \(\Gamma\)]
  The \textbf{forgetful functor}
  \[
    \Gamma \colon \Asm{\AA} \to \Set
  \]
  is defined by taking an assembly \(X\) to its underlying set \(\carrier{X}\)
  and a morphism \(f \colon X \to Y\) of assemblies to the map of sets
  \(f \colon \carrier{X} \to \carrier{Y}\).
\end{definition}

\begin{exercise}\label{exer:Gamma-global-sections}
  Show that \(\Gamma\) is naturally isomorphic to the \textbf{global sections}
  functor:
  \begin{align*}
    \Asm{\AA} &\to \Set \\
    X &\mapsto \Asm{\AA}(\One,X) \\
    f \colon X \to Y &\mapsto \text{post-composition with \(f\)}
  \end{align*}
\end{exercise}

To get a functor \(\nabla \colon \Set \to \Asm{\AA}\) we need a procedure for
turning a set \(X\) into an assembly \(\nabla X\).
%
Of course, it makes sense to let \(\carrier{\nabla(X)}\) be the set \(X\), but
what should the realizers be?
%
In general, there is no reason why an element \(x\) of an arbitary set \(X\)
should have an ``implemention'' in the pca \(\AA\).
%
In light of this, one might be tempted to say that there are no realizers of
\(x\). This is not a good idea for two reasons:
\begin{enumerate}[(1)]
\item It does not define an assembly, because, by definition, every element should
  have at least one realizer.
\item Even if the definition of assembly did allow for an empty set of
  realizers, then the tracking requirement on assembly maps would mean that
  there are no assembly maps \(\Two \to \nabla\set{0,1}\) which does not make
  sense if \(\nabla\set{0,1}\) is an assembly with no computational data.
\end{enumerate}

The solution is to say that \emph{all} elements of \(\AA\) are actually realizers
of \(x \in \carrier{\nabla(X)}\).
%
The idea is that \(\nabla(X)\) carries no meaningful computational data, because
all elements have the same set of realizers, so from this perspective all
elements of the assembly look alike and computationally speaking we can't tell
them apart.


\begin{definition}[\(\nabla\)]
  We define a functor
  \[
    \nabla \colon \Set \to \Asm{\AA}
  \]
  by mapping a set \(X\) to the assembly with carrier \(X\) and
  \(\pca{a} \realizes_X x\) for \emph{all} elements \(\pca{a} \in \AA\).
  %
  A map of sets \(f \colon X \to Y\) gets send to \(f\) and is tracked by
  \(\icomb\).
\end{definition}

\begin{exercise}\label{exer:Gamma-left-adjoint-to-nabla}
  Prove that \(\Gamma\) is left adjoint to \(\nabla\).
\end{exercise}

\begin{notation}[\(\eta\)]
  We write \(\eta\) for the unit of the adjunction \(\Gamma \vdash \nabla\),
  i.e.\ for each assembly \(X\), we have an assembly map
  \[
    \eta_X \colon X \to \nabla \carrier{X}
  \]
  given by the identity on \(\carrier{X}\) and tracked by (for example)
  \(\icomb\).
\end{notation}

In particular, we have an assembly map
\(\nabla_\Two \colon \Two \to \nabla\set{0,1}\).
%
We do not expect a map in the reverse direction as it would mean that we can
compute the relevant boolean (\(\pcatrue\) or \(\pcafalse\)) without receiving
any relevant computational input. Indeed, we have:

\begin{exercise}\label{exer:no-nabla-to-Two}
  Show that there are no assembly maps \(f \colon \nabla\set{0,1} \to \Two\) in
  \(\Asm{\AA}\) unless the pca \(\AA\) is trivial.

  \emph{Hint}: Use~\cref{exer:nontrivial-pca}.
\end{exercise}

To complete the section, the following exercises ask you to check that the
functors \(\Gamma\) and \(\nabla\) do not have other adjoints.

\begin{exercise}\label{exer:nabla-no-right-adjoint}
  Show that \(\nabla\) does not have a right adjoint when \(\AA\) is nontrivial.

  % \emph{Hint}: Consider the assemblies \(\Two\) (recall~\cref{TODO}) and \(\nabla\set{0,1}\).
\end{exercise}

\begin{exercise}[cf.~{\cite[Lemma~5.1.7]{Zoethout2018}}]\label{exer:Gamma-no-left-adjoint}
  We assume that the pca \(\AA\) is nontrivial. The aim of these exercises is to
  conclude that \(\Gamma\) does not have a left adjoint.
  \begin{enumerate}[(i)]
  \item Show that, for any object \(X \in \Asm{\AA}\), there are at most
    \(\carrier{\AA}\)-many arrows from \(X\) to \(\Two\).
  \item Use the above to prove that the \(\AA\)-indexed coproduct of copies of
    \(\One\) does not exist in \(\Asm{\AA}\).
  \item Conclude that \(\Gamma\) does not have a left adjoint.
  \end{enumerate}
\end{exercise}

The functor \(\nabla\) plays an important role in classifying regular
monomorphisms and recovering classical logic within realizability logic as
explained in \cref{chap:logic}.

\section{Epimorphisms and monomorphisms}
Epimorphisms and monomorphisms and their regular counterparts will be very
important to the logical side of the category of assemblies (see
\cref{chap:logic}), but studying them also provides excellent opportunities
for improving our understanding and intuition of assemblies in general.

\begin{proposition}[Characterization of epis and monos]
  For an assembly map \(f\), we have the following equivalences:
  \begin{enumerate}[(i)]
  \item \(f \colon X \to Y\) is an epimorphism if and only if
    \(f \colon \carrier{X} \to \carrier{Y}\) is surjective;
  \item \(f \colon X \to Y\) is a monomorphism if and only if
    \(f \colon \carrier{X} \to \carrier{Y}\) is injective.
  \end{enumerate}
\end{proposition}
\begin{proof}
  Surjectivity is clearly sufficient to force an assembly map to be epi.
  %
  For the converse, we use that \(\Gamma\) preserves epimorphisms as it is a
  left adjoint%
  \footnote{In any category, a morphism \(f\) is an epi if and only if the square
    \begin{tikzcd}[ampersand replacement=\&,column sep=3mm,row sep=3mm]
      X \ar[r,"f"] \ar[d,"f"'] \& Y \ar[d,"\id"] \\
      Y \ar[r,"\id"] \& Y
    \end{tikzcd}
    is a pushout. Since left adjoints preserve colimits (and identities), they
    also preserve epimorphisms\label{epi-mono-preservation}.}.
  %
  Thus, if \(f \colon \carrier{X} \to \carrier{Y}\) is an epi, then
  \(\Gamma(f)\) must be an epi in \(\Set\), i.e. a surjection.

  For the characterization of monomorphisms, injectivity is again clearly
  sufficient.
  %
  Conversely, it follows from our construction of products and equalizers that
  \(\Gamma\) preserves monomorphisms (use the dual
  of~\footref{epi-mono-preservation}).
  % \footnote{Use the dual
  %  of~\footref{epi-mono-preservation}.}.
  %
  Thus, if \(f \colon \carrier{X} \to \carrier{Y}\) is a mono, then
  \(\Gamma(f)\) must be a mono in \(\Set\), i.e. an injection.
\end{proof}

At first sight, it is perhaps surprising that being an epimorphism or
monomorphism depends solely on the underlying function of an assembly map with
the realizers playing no role.
%
The situation is very different for \emph{regular} epimorphisms and
monomorphisms.

\subsection{Regular epimorphisms}\label{sec:regular-epis}
Recall that a morphism \(f \colon X \to Y\) is a \textbf{regular epimorphism}
if it fits in a coequalizer diagram
\[
  \begin{tikzcd}
    Z \ar[r,shift left, "g"]\ar[r, shift right, "h"']
    & X \ar[r,"f"]
    & Y
  \end{tikzcd}
\]
for some morphisms \(g\) and \(h\).


\begin{exercise}[Characterization of regular epimorphisms]%
  \label{exer:characterize-regular-epis}
  Prove that an assembly map \(f \colon X \to Y\) is a regular epimorphism if
  and only if \(f \colon \carrier{X} \to \carrier{Y}\) is surjective and there
  exists an element \(\pca{s} \in \AA\) such that for all \(y \in \carrier{Y}\)
  and \(b \realizes_Y y\), we have \(\pca{s}\pca{b} \realizes_X x\) for some
  \(x \in \carrier{X}\) with \(f(x) = y\).
\end{exercise}

The element \(\pca{s}\) in \cref{exer:characterize-regular-epis} effectively
witnesses the surjectivity of \(f\).

\begin{exercise}\label{exer:epi-but-not-regular-epi}
  Give an example of an epimorphism in \(\Asm{\AA}\) which is not regular (for a
  nontrivial pca \(\AA\)).
\end{exercise}

\begin{proposition}\label{regular-epi-pullback-stable}
  The regular epimorphisms in \(\Asm{\AA}\) are stable under pullback along
  arbitrary assembly maps.
\end{proposition}
\begin{proof}
  Consider a pullback diagram
  \[
    \begin{tikzcd}
      X \times_Z Y \pbcorner
      \ar[r,"\pi_2"]
      \ar[d,"\pi_1"']
      & Y \ar[d,"g"] \\
      X\ar[r,"f"] & Z
    \end{tikzcd}
  \]
  with \(g\) a regular epimorphism. We must show that \(\pi_1\) is also a
  regular epi.
  %
  From the description of equalizers and products we can compute that
  \begin{align*}
    &\carrier{X \times_Z Y} \coloneq \set{(x,y) \mid f(x) = g(y)}
    \text{ with realizers}\\
    &\pcapair\pca{a}\pca{b} \realizes_{X \times_Z Y} (x,y)
    \text{ for }
    \pca{a} \realizes_X x
    \text{ and }
    \pca{b} \realizes_Y y.
  \end{align*}
  By assumption and~\cref{exer:characterize-regular-epis} there exists an
  element \(s \in \AA\) such that for every \(z \in \carrier{Z}\) and
  \(\pca{c} \realizes_Z z\) we have \(\pca{s}\pca{c} \realizes_Y y\) for some
  \(y \in \carrier{Y}\) with \(g(y) = z\).
  %
  Now if \(\pca{t}\) tracks \(f\) and we put
  \[
    \pca{s'} \coloneq \lambdapca{u}{\pcapair u\,(\pca{s}(\pca{t}u))},
  \]
  then for every \(x \in \carrier{X}\) and \(\pca{a} \realizes_X x\) we have
  \(\pca{s'}\pca{a} \realizes_{X \times_Z Y} (x,y)\) for some
  \(y \in \carrier{Y}\) with \(g(y) = f(x)\).
  %
  Hence, \(\pi_1\) is a regular epi by~\cref{exer:characterize-regular-epis}, as
  desired.
\end{proof}

The importance of \cref{regular-epi-pullback-stable} is that, together with
\cref{base-change-adjoints}, it gives the category of assemblies enough
structure to interpret first order logic as we explore in the next chapter.

We recall that in the category of sets, every function \(f \colon A \to B\)
factors as a regular epimorphism (= surjection) followed by a monomorphism (=
injection):
\[
  \begin{tikzcd}[row sep=3mm,column sep=3mm]
    A \ar[dr,"\tilde f"'] \ar[rr,"f"] & & B \\
    & \im(f) \ar[ur,hookrightarrow]
  \end{tikzcd}
\]
where \(\im(f) \coloneq \set{b \in B \mid \exists (a \in A).f(a) = b}\).
%

The same is true in any category with finite limits and pullback stable regular
epimorphisms.
%
For the category of assemblies it is also straightforward to calculate the
factorization directly, as we ask you to verify:

\begin{exercise}\label{exer:reg-epi-mono-factorization}
  Given an assembly map \(f \colon X \to Y\), show how to factor it in
  \(\Asm{\AA}\) as a regular epimorphism followed by a monomorphism.
\end{exercise}

\subsection{Regular monomorphisms}\label{sec:regular-monos}
There is also a nice characterization of regular monomorphisms which,
as we explain in the next chapter, represent classical (or more precisely,
\(\lnot\lnot\)-stable) subsets of assemblies.

Recall that a morphism \(f \colon X \to Y\) is a \textbf{regular monomorphism}
if it fits in an equalizer diagram
\[
  \begin{tikzcd}
    X \ar[r,"f"]
    & Y \ar[r,shift left, "g"]\ar[r, shift right, "h"']
    & Z
  \end{tikzcd}
\]
for some morphisms \(g\) and \(h\).

\begin{exercise}[Characterization of regular monomorphisms]%
  \label{exer:characterize-regular-monos}
  Prove that an assembly map \(f \colon X \to Y\) is a regular monomorphism if
  and only if \(f \colon \carrier{X} \to \carrier{Y}\) is injective and there
  exists an element \(\pca{i} \in \AA\) such that
  \(\pca{i}\pca{b} \realizes_X x\) for all \(x \in \carrier{X}\) and
  \(\pca{b} \realizes_Y f(x)\).
\end{exercise}

Notice that the element \(\pca{i}\) in \cref{exer:characterize-regular-monos}
acts like an effective witness of left-cancellability of \(f\): from a realizer
of \(f(x)\) we can effectively find a realizer of \(x\).

\begin{exercise}\label{exer:mono-but-not-regular-mono}
  Give an example of a monomorphism in \(\Asm{\AA}\) which is not regular (for a
  nontrivial pca \(\AA\)).
\end{exercise}


\section{From pcas to assemblies, functorially}
In this chapter we constructed a category of assemblies \(\Asm{\AA}\) over an
arbitrary pca \(\AA\). The construction \(\AA \mapsto \Asm{\AA}\) is actually
functorial in a suitable sense.
%
To make this precise, we need a \emph{category} of pcas.
%
On first thought, one might think that a suitable notion of a morphism between
pcas \(\AA\) and \(\BB\) is a map on the underlying sets of the pcas that
preserves application.
%
This turns out \emph{not} to be an appropriate notion of morphism however. It
would be if a pca should be thought of as an algebraic gadget, but it shouldn't:
for example, the application map is associative if and only if the pca is
trivial~\cite[Proposition~1.3.1]{vanOosten2008}.
%

Instead, an appropriate notion of morphism, due to Longley~\cite{Longley1995},
is that of an \emph{applicative morphism} (re-branded to \emph{simulation}
in~\cite{Bauer2023}).
%
For lack of space, we will not go into the details here and instead give some
relevant pointers to the literature.

The intuition is that a morphism between pcas from \(\AA\) to \(\BB\) should
assign to each program \(\pca{a} \in \AA\) one or more programs in \(\BB\) that
\emph{simulate} \(\pca{a}\) in \(\BB\).
%
Moreover, similar to the requirement that assembly maps are tracked, the
simulation should have an effective witness in the pca \(\BB\).
%
With applicative morphisms between them we get a category of pcas. In fact, this
category is preorder-enriched, so we even get a 2-category.

Now, one can show that every applicative morphism \(\gamma \colon \AA \to \BB\)
gives rise to a functor \(F_\gamma \colon \Asm{\AA} \to \Asm{\BB}\).
%
The induced functor \(F_\gamma\) is regular (i.e.\ it preserves finite limits
and regular epis) and is a so-called \emph{\(S\)-functor}: it is the
identity on underlying sets and functions.
%
In fact, every such functor arises from an applicative morphism.
%
In the end, we actually get an equivalence of
2-categories~\cite[Theorem~1.6.2]{vanOosten2008} between:
\begin{itemize}
\item the 2-category of pcas with applicative morphisms and their preorders, and
\item the 2-category of categories of assemblies with regular \(S\)-functors between
  them and (necessarily unique) natural transformations between the functors.
\end{itemize}




\section{List of exercises}
\begin{enumerate}
\item \cref{exer:coproducts}: On the universal property of coproducts.
\item \cref{exer:coproduct-booleans}: On the assembly of booleans as the
  coproduct \(\One + \One\).
\item \cref{exer:nno}: On natural numbers objects.
\item \cref{exer:Gamma-global-sections}: On the forgetful functor \(\Gamma\) and
  the global sections functor.
\item \cref{exer:Gamma-global-sections}: On \(\Gamma\) being a left adjoint to
  \(\nabla\).
\item \cref{exer:no-nabla-to-Two}: On the nonexistence of assembly maps
  \(\nabla\set{0,1} \to \Two\).
\item \cref{exer:nabla-no-right-adjoint}: On the nonexistence of a right
  adjoint to \(\nabla\).
\item \cref{exer:Gamma-no-left-adjoint}: On the nonexistence of a left
  adjoint to \(\Gamma\).
\item \cref{exer:characterize-regular-epis}: On characterizing the regular
  epimorphisms.
\item \cref{exer:epi-but-not-regular-epi}: On an example of an epimorphism
  which is not a regular.
\item \cref{exer:reg-epi-mono-factorization}: On factoring an assembly map as a
  regular epi followed by a mono.
\item \cref{exer:characterize-regular-monos}: On characterizing the regular
  monomorphisms.
\item \cref{exer:mono-but-not-regular-mono}: On an example of a monomorphism
  which is not a regular.
\end{enumerate}

%%% Local Variables:
%%% mode: latexmk
%%% TeX-master: "../main"
%%% End:
