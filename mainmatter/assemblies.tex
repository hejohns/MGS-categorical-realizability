\chapter{Categories of assemblies}\label{chap:assemblies}


\begin{definition}[Assembly]
  An \textbf{assembly} over a pca \(\AA\) is a set \(X\) together with a
  relation \({\realizes}\) between \(\AA\) and \(X\) such that for all
  \(x \in X\), there exists at least one element \(\pca{a} \in \AA\) with
  \(\pca{a} \realizes x\).
\end{definition}

The relation \(\pca{a} \realizes x\) is pronounced as ``\(\pca{a}\)
\textbf{realizes} \(x\)'' and we also say that \(\pca{a}\) is a
\textbf{realizer} of \(x\). We think of \(\pca{a}\) as an \emph{implementation}
of \(x \in X\) in the pca.
%
The requirement on assemblies is that each element of the set must have at least
one implementation.

\begin{notation}[\(\carrier{X}\), \({\realizes_X}\)]
  Given an assembly \(X\), we will write \(\carrier{X}\) for its underlying set
  and \(\realizes_X\) for its relation between \(\AA\) and \(\carrier{X}\).
\end{notation}

\begin{example}[The assembly of booleans, \(\Two\)]\label{ex:assembly-of-booleans}
  The \textbf{assembly of booleans}, denoted by \(\Two\), is defined as
  \[
    \carrier{\Two} \coloneqq \set{0,1},
    \quad\quad
    \pcafalse \realizes_\Two 0
    \quad\quad
    \text{and}\quad\quad
    \pcatrue \realizes_\Two 1,
  \]
  where we recall the booleans \(\pcafalse\) and \(\pcatrue\) from \cref{sec:booleans}.
\end{example}

\begin{example}[The assembly of natural numbers, \(\NatAsm\)]\label{ex:NatAsm}
  The \textbf{assembly of natural numbers}, denoted by \(\NatAsm\), is defined as
  \[
    \carrier{\NatAsm} \coloneqq \Nat
    \quad\quad\text{and}\quad\quad
    \numeral{n} \realizes_\NatAsm n \text{ for each \(n \in \Nat\)},
  \]
  where we recall that \(\numeral{n}\) is the \(n\)\textsuperscript{th} Curry
  numeral from \cref{sec:numerals}.
\end{example}

\begin{example}
  Taking \(\AA = \Kone\), we can consider the assembly \(X\) of Turing computable functions:
  \[
    \carrier{X} \coloneqq \set{f \colon \Nat \to \Nat \mid f \text{ is Turing computable}}
    \quad\quad\text{and}\quad\quad
    m \realizes_X f \iff \prenum{m} = f
  \]
  where we recall \(\Kone\) and \(\prenum{-}\) from~\cref{ex:Kleene-1}.
  %
  We remark that each \(f \in \carrier{X}\) has infinitely many realizers.

  Notice that, with this realizability relation, we cannot let \(\carrier{X}\)
  be the set of \emph{all} functions from \(\Nat\) to \(\Nat\), because then the
  set of realizers of a noncomputable function would be empty, which is not
  allowed by the definition of an assembly.
\end{example}

\section{Morphisms of assemblies}

\begin{definition}[Track]
  For assemblies \(X\) and \(Y\), we say that an element \(\pca{t} \in \AA\)
  \textbf{tracks} a function \(f \colon \carrier{X} \to \carrier{Y}\) if for all
  \(x \in \carrier{X}\) and \(\pca{a} \in \AA\), if \(\pca{a} \realizes_X x\), then
  \(\pca{t}\pca{a}\) is defined and \(\pca{t}\pca{a} \realizes_Y f(x)\).
\end{definition}

\begin{notation}
  We will shorten the above to: ``\(\pca{t}\pca{a} \realizes_Y f(x)\)
  for all \(x \in \carrier{X}\) and \(\pca{a} \realizes_X x\)''.
  %
  That is, we implicitly quantify over \(\pca{a}\) and we implicitly assume that
  \(\pca{t}\pca{a}\) is defined when we write
  \(\pca{t}\pca{a} \realizes_Y f(x)\).
\end{notation}

\begin{definition}[Assembly map]
  An \textbf{assembly map} from an assembly \(X\) to \(Y\) is a function
  \(f \colon \carrier{X} \to \carrier{Y}\) that is tracked by some element.
  %
  The existence of a tracker is a required \emph{property} of the morphism and
  \emph{not} part of the data.
\end{definition}

\begin{proposition}
  Assemblies and assembly maps form a category with composition given
  by composition of functions on underlying sets.
\end{proposition}
\begin{proof}
  We need to verify that composition is well defined, i.e., that if
  \(f \colon X \to Y\) and \(g \colon Y \to Z\) are assembly maps, then
  \(g \circ f \colon \carrier{X} \to \carrier{Z}\) is tracked.
  %
  Let \(\pca{t_f}\) and \(\pca{t_g}\) track \(f\)~and~\(g\), respectively.  We
  claim that \(\lambdapca{x}{\pca{t_g}(\pca{t_f}(x))}\) tracks \(g \circ
  f\). Indeed, the closed term \(\lambdapca{x}{\pca{t_g}(\pca{t_f}(x))}\) is
  defined by construction, and if \(\pca{a} \realizes_X x\), then
  \[
    {(\lambdapca{x}{\pca{t_g}(\pca{t_f}(x)}))\pca{a} =
    \pca{t_g}(\pca{t_f}\pca{a})} \realizes_Z g(f(x))
  \]
  by choice of \(\pca{t_f}\) and \(\pca{t_g}\).
  %
  Moreover, for each assembly \(X\), we have an identity morphism on \(X\) given
  by the identity on \(\carrier{X}\) and tracked by \(\icomb\).
  %
  Finally, associativity of composition holds because composing functions of
  sets is associative.
\end{proof}

\begin{notation}[\(\Asm{\AA}\)]
  We write \(\Asm{\AA}\) for the category of assemblies over a pca \(\AA\).
\end{notation}

\begin{remark}
  \textcolor{Mulberry}{TODO: Relative category of assemblies + slogans (Bauer)}
\end{remark}

\section{Categorical constructions}

\subsection{Cartesian closure and equalizers}
\begin{proposition}[Terminal object]
  The terminal object \(\One\) in \(\Asm{\AA}\) is given by
  \[
    \carrier{\Zero} \coloneq \set{\singleton}
    \quad
    \text{and}
    \quad
    \pca{a} \realizes_\One \singleton
    \text{ for all } \pca{a} \in \AA.
  \]
\end{proposition}
\begin{proof}
  As in \(\Set\).
\end{proof}


\begin{proposition}[Products]
  The product \(X \times Y\) of two assemblies \(X\)~and~\(Y\) is given by
  \[
    \carrier{X \times Y} \coloneq \carrier{X} \times \carrier{Y}
    \quad\text{and}\quad
    \pcapair\pca{a}\pca{b} \realizes_{X \times Y} (x,y)
    \text{ for }
    \pca{a} \realizes_X x
    \text{ and }
    \pca{b} \realizes_Y y.
  \]
\end{proposition}
\begin{proof}
  The projection maps \(\pi_1 \colon X \times Y \to X\) and
  \(\pi_2 \colon X \times Y \to Y\) are given by \((x,y) \mapsto x\) and
  \((x,y) \mapsto y\), and tracked by \(\pcafst\) and \(\pcasnd\), respectively.
  %
  Moreover, every pair of assembly maps \(f \colon Z \to X\) and
  \(g \colon Z \to Y\) induces an assembly map
  \(\langle f,g\rangle \colon Z \to X \times Y\) given by
  \(z \mapsto (f(z),g(z))\) and tracked by
  \(\lambdapca{u}{\pcapair(\pca{t_f}u)(\pca{t_g}u)}\) when \(\pca{t_f}\) and
  \(\pca{t_g}\) track \(f\) and \(g\), respectively.
\end{proof}

\textcolor{Mulberry}{TODO: Explain}
\begin{proposition}[Exponentials]
  The exponential \(Y^X\) of two assemblies \(X\)~and~\(Y\) is given by
  \[
    \carrier*{Y^X} \coloneq
    \text{the set of assembly maps from \(X\) to \(Y\)}
    \quad\text{and}\quad
    \pca{t} \realizes_{Y^X} f
    \text{ if \(\pca{t}\) tracks \(f\)}.
  \]
\end{proposition}
\begin{proof}
  The evaluation morphism \(\operatorname{ev} \colon Y^X \times X \to Y\) given
  by \((f,x) \mapsto f(x)\) is tracked by
  \(\lambdapca{u}{\pcafst u(\pcasnd u)}\).
  %
  Moreover, every \(g \colon Z \times X \to Y\) induces a unique assembly map
  \(\tilde g \colon Z \to Y^X\) making the diagram
  \[
    \begin{tikzcd}
      Y^X \times X \ar[rr,"\operatorname{ev}"]
      & & Y \\
      & Z \times X \ar[ur,"g"']
      \ar[ul,"{\tilde g} \,\times\, {\id_X}"]
    \end{tikzcd}
  \]
  commute.
  %
  Indeed, there is a unique assignment
  \(\tilde g(z) \coloneq (x \mapsto g(z,x))\) and this assignment is tracked by
  \(\lambdapca{u}{(\lambdapca{v}{\pca{t_g}(\pcapair u\,v)})}\) when
  \(\pca{t_g}\) tracks \(g\).
\end{proof}

Thus, we conclude that \(\Asm{\AA}\) is cartesian closed.

\begin{proposition}[Equalizers]
  The equalizer \(E\) of two assembly maps \(f,g \colon X \to Y\) is given by
  \[
    \carrier{E} \coloneq \set{x \in \carrier{X} \mid f(x) = g(x)}
    \quad\text{and}\quad
    \pca{a} \realizes_E x \text{ if } \pca{a} \realizes_X x.
  \]
\end{proposition}
\begin{proof}
  On the level of underlying sets, this is as in the category of sets, so it
  suffices to show that the relevant functions are tracked.
  %
  The inclusion \(i \colon \carrier{E} \to \carrier{X}\) is tracked by
  \(\icomb\).
  %
  Given an assembly map \(h \colon D \to X\) such that
  \(f \circ h = f \circ h\), the map \(h \colon \carrier{D} \to \carrier{X}\)
  factors uniquely through \(i\) via \(k \colon \carrier{D} \to \carrier{E}\)
  which is tracked by any tracker of \(h\).
\end{proof}

\subsection{Colimits}
%
We now move on to colimits in the category of assemblies.
\textcolor{Mulberry}{TODO: Explain and refs}

\begin{proposition}[Initial object]
  The initial object \(\Zero\) in \(\Asm{\AA}\) is given by
  \(
    \carrier{\Zero} \coloneq \emptyset.
  \)
\end{proposition}
\begin{proof}
  As in \(\Set\).
\end{proof}

\begin{proposition}[Coproducts]\label{coproducts}
  The coproduct \(X + Y\) of two assemblies \(X\) and \(Y\) is given by
  \begin{align*}
    \carrier{X + Y} \coloneq \carrier{X} + \carrier{Y}
    \quad\text{and}\quad
    &\pcapair\pcafalse\pca{a} \realizes_{X + Y} \inl(x)
      \text{ for } \pca{a} \realizes_X x;
    \\
    &\pcapair\pcatrue\pca{b} \realizes_{X + Y} \inr(y)
    \text{ for }
    \pca{b} \realizes_Y y.
  \end{align*}
\end{proposition}
\begin{exercise}\label{exer:coproducts}
  Prove~\cref{coproducts}.

  \emph{Warning}: Carefully check that the closed terms you give for the
  trackers are defined and thus give elements of the pca as required.
\end{exercise}

\begin{exercise}\label{exer:coproduct-booleans}
  Show that \({1 + 1} \cong \Two\), where \(\Two\) is the assembly of booleans
  defined in~\cref{ex:assembly-of-booleans}.
\end{exercise}


\begin{proposition}[Coequalizers]
  The coequalizer \(C\) of assembly maps \(f,g \colon X \to Y\) is given by
  \[
    \carrier{C} \coloneq \carrier{Y}/{\sim}
    \quad\text{and}\quad
    \pca{a} \realizes_C [y] \text{ if } \pca{a} \realizes_Y y'
    \text{ for some } {y'} \sim y,
  \]
  where \({\sim}\) is the least equivalence relation on \(\carrier{Y}\)
  generated by \(f(x) \sim g(x)\) for all \(x \in \carrier{X}\).
  % setting \(y \sim y'\) whenever there exists \(x \in \carrier{X}\)
  % with \(f(x) = y\) and \(g(x) = y'\).
\end{proposition}
\begin{proof}
  On the level of underlying sets, this is as in the category of sets, so it
  suffices to show that the relevant functions are tracked.
  %
  The quotient map \(q \colon \carrier{Y} \to \carrier{Y}/{\sim}\) is tracked by
  \(\icomb\).
  %
  Given an assembly map \(h \colon Y \to Z\) such that
  \(h \circ f = h \circ g\), the map \(h \colon \carrier{Y} \to \carrier{Z}\)
  factors uniquely through \(q\) via
  \(k \colon \carrier{Y}/{\sim} \to \carrier{Z}\) with \(k([y]) \coloneq h(y)\).
  %
  Moreover, the function \(k\) is tracked by any tracker of \(h\).
\end{proof}

\subsection{Natural numbers object}
The notion of natural numbers can be captured via a universal
property which is due to Lawvere~\cite{Lawvere1963}. In a category
\(\mathcal C\), a \textbf{natural numbers object (nno)} is an object \(N\) equipped
with morphisms \(z \colon 1 \to N\) (``zero'') and \(s \colon N \to N\)
(``successor'') such that for all triples
\((X,x \colon 1 \to X,f\colon X \to X)\) there is a unique morphism
\(r \colon N \to X\) (defined by ``recursion'') making the diagram
\[
  \begin{tikzcd}[row sep=12mm,column sep=12mm]
    & N \ar[r,"s"] \ar[d,dashed,"r"] & N \ar[d,dashed,"r"] \\
    1 \ar[ur,"z"] \ar[r,"x"] & X \ar[r,"f"] & X
  \end{tikzcd}
\]
commute.

\begin{exercise}\label{exer:nno} \leavevmode
  \begin{enumerate}[(i)]
  \item Exhibit \(\Nat\) as a nno in \(\Set\).
  \item Exhibit \(\NatAsm\) (from~\cref{ex:NatAsm}) as a nno in \(\Asm{\AA}\).

    \emph{Hint}: Recall~\cref{exer:primitive-recursion}.
  \end{enumerate}
\end{exercise}

\section{Relation to the category of sets}

\begin{definition}[Forgetful functor, \(\Gamma\)]
  The \textbf{forgetful functor}
  \[
    \Gamma \colon \Asm{\AA} \to \Set
  \]
  is defined by taking an assembly \(X\) to its underlying set \(\carrier{X}\)
  and a morphism \(f \colon X \to Y\) of assemblies to the map of sets
  \(f \colon \carrier{X} \to \carrier{Y}\).
\end{definition}

\begin{exercise}\label{exer:Gamma-global-sections}
  Show that \(\Gamma\) is naturally isomorphic to the \textbf{global sections}
  functor:
  \begin{align*}
    \Asm{\AA} &\to \Set \\
    X &\mapsto \Asm{\AA}(\One,X) \\
    f \colon X \to Y &\mapsto \text{post-composition with \(f\)}
  \end{align*}
\end{exercise}

\textcolor{Mulberry}{TODO: Explain intuition for \(\nabla\). Taking an empty set
  of realizers (which is prohibited anyway) would yield a map
  \(\nabla\set{0,1} \to \Two\) which is undesirable.}
\begin{definition}[\(\nabla\)]
  We define a functor
  \[
    \nabla \colon \Set \to \Asm{\AA}
  \]
  by mapping a set \(X\) to the assembly with carrier \(X\) and
  \(\pca{a} \realizes_X x\) for \emph{any} element \(\pca{a} \in \AA\).
  %
  A map of sets \(f \colon X \to Y\) gets send to \(f\) and is tracked by
  \(\icomb\).
\end{definition}

\begin{exercise}\label{exer:Gamma-left-adjoint-to-nabla}
  Prove that \(\Gamma\) is left adjoint to \(\nabla\).
\end{exercise}

\begin{notation}[\(\eta\)]
  We write \(\eta\) for the unit of the adjunction \(\Gamma \vdash \nabla\),
  i.e.\ for each assembly \(X\), we have an assembly map
  \[
    \eta_X \colon X \to \nabla \carrier{X}
  \]
  given by the identity on \(\carrier{X}\) and tracked by (for example)
  \(\icomb\).
\end{notation}

\begin{exercise}\label{exer:no-nabla-to-Two}
  Show that there are no assembly maps \(f \colon \nabla\set{0,1} \to \Two\) unless
  \(\AA\) is trivial.

  \emph{Hint}: Use~\cref{exer:nontrivial-pca}.
\end{exercise}

\begin{exercise}\label{exer:nabla-no-right-adjoint}
  Show that \(\nabla\) does not have a right adjoint when \(\AA\) is nontrivial.

  % \emph{Hint}: Consider the assemblies \(\Two\) (recall~\cref{TODO}) and \(\nabla\set{0,1}\).
\end{exercise}

\begin{exercise}[cf.~{\cite[Lemma~5.1.7]{Zoethout2018}}]\label{exer:Gamma-no-left-adjoint}
  We assume that the pca \(\AA\) is nontrivial. The aim of these exercises is to
  conclude that \(\Gamma\) does not have a left adjoint.
  \begin{enumerate}[(i)]
  \item Show that, for any object \(X \in \Asm{\AA}\), there are at most
    \(\carrier{\AA}\)-many arrows from \(X\) to \(\Two\).
  \item Use the above to prove that the \(\AA\)-indexed coproduct of copies of
    \(\One\) does not exist in \(\Asm{\AA}\).
  \item Conclude that \(\Gamma\) does not have a left adjoint.
  \end{enumerate}
\end{exercise}

\section{Epimorphisms and monomorphisms}
\textcolor{Mulberry}{TODO: Intro on epis and monos}

\begin{proposition}[Characterization of epis and monos]
  For an assembly map \(f\), we have the following equivalences:
  \begin{enumerate}[(i)]
  \item \(f \colon X \to Y\) is an epimorphism if and only if
    \(f \colon \carrier{X} \to \carrier{Y}\) is surjective;
  \item \(f \colon X \to Y\) is a monomorphism if and only if
    \(f \colon \carrier{X} \to \carrier{Y}\) is injective.
  \end{enumerate}
\end{proposition}
\begin{proof}
  Surjectivity is clearly sufficient to force an assembly map to be epi.
  %
  For the converse, we use that \(\Gamma\) preserves epimorphisms as it is a
  left adjoint%
  \footnote{In any category, a morphism \(f\) is an epi if and only if the square
    \begin{tikzcd}[ampersand replacement=\&,column sep=3mm,row sep=3mm]
      X \ar[r,"f"] \ar[d,"f"'] \& Y \ar[d,"\id"] \\
      Y \ar[r,"\id"] \& Y
    \end{tikzcd}
    is a pushout. Since left adjoints preserve colimits (and identities), they
    also preserve epimorphisms\label{epi-mono-preservation}.}.
  %
  Thus, if \(f \colon \carrier{X} \to \carrier{Y}\) is an epi, then
  \(\Gamma(f)\) must be an epi in \(\Set\), i.e. a surjection.

  For the characterization of monomorphisms, injectivity is again clearly
  sufficient.
  %
  Conversely, it follows from our construction of products and equalizers that
  \(\Gamma\) preserves monomorphisms\footnote{Use the dual
    of~\footref{epi-mono-preservation}.}.
  %
  Thus, if \(f \colon \carrier{X} \to \carrier{Y}\) is a mono, then
  \(\Gamma(f)\) must be a mono in \(\Set\), i.e. an injection.
\end{proof}

\subsection{Regular epimorphisms}
Recall that a morphism \(f \colon X \to Y\) is a \textbf{regular epimorphism}
if it fits in a coequalizer diagram
\[
  \begin{tikzcd}
    Z \ar[r,shift left, "g"]\ar[r, shift right, "h"']
    & X \ar[r,"f"]
    & Y
  \end{tikzcd}
\]
for some morphisms \(g\) and \(h\).


\begin{exercise}[Characterization of regular epimorphisms]%
  \label{exer:characterize-regular-epis}
  Prove that an assembly map \(f \colon X \to Y\) is a regular epimorphism if
  and only if \(f \colon \carrier{X} \to \carrier{Y}\) is surjective and there
  exists an element \(\pca{s} \in \AA\) such that for all \(y \in \carrier{Y}\)
  and \(b \realizes_Y y\), we have \(\pca{s}\pca{b} \realizes_X x\) for some
  \(x \in \carrier{X}\) with \(f(x) = y\).
\end{exercise}

\begin{exercise}\label{exer:epi-but-not-regular-epi}
  Give an example of an epimorphism in \(\Asm{\AA}\) which is not regular (for a
  nontrivial pca \(\AA\)).
\end{exercise}

\begin{proposition}
  The regular epimorphisms are stable under pullback.
\end{proposition}
\begin{proof}
  Consider a pullback diagram
  \[
    \begin{tikzcd}
      X \times_Z Y \pbcorner
      \ar[r,"\pi_2"]
      \ar[d,"\pi_1"']
      & Y \ar[d,"g"] \\
      X\ar[r,"f"] & Z
    \end{tikzcd}
  \]
  with \(g\) a regular epimorphism. We must show that \(\pi_1\) is also a
  regular epi.
  %
  From the description of equalizers and products we can compute that
  \begin{align*}
    &\carrier{X \times_Z Y} \coloneq \set{(x,y) \mid f(x) = g(y)}
    \text{ with realizers}\\
    &\pcapair\pca{a}\pca{b} \realizes_{X \times_Z Y} (x,y)
    \text{ for }
    \pca{a} \realizes_X x
    \text{ and }
    \pca{b} \realizes_Y y.
  \end{align*}
  By assumption and~\cref{exer:characterize-regular-epis} there exists an
  element \(s \in \AA\) such that for every \(z \in \carrier{Z}\) and
  \(\pca{c} \realizes_Z z\) we have \(\pca{s}\pca{c} \realizes_Y y\) for some
  \(y \in \carrier{Y}\) with \(g(y) = z\).
  %
  Now if \(\pca{t}\) tracks \(f\) and we put
  \[
    \pca{s'} \coloneq \lambdapca{u}{\pcapair u\,(\pca{s}(\pca{t}u))},
  \]
  then for every \(x \in \carrier{X}\) and \(\pca{a} \realizes_X x\) we have
  \(\pca{s'}\pca{a} \realizes_{X \times_Z Y} (x,y)\) for some
  \(y \in \carrier{Y}\) with \(g(y) = f(x)\).
  %
  Hence, \(\pi_1\) is a regular epi by~\cref{exer:characterize-regular-epis}, as
  desired.
\end{proof}

\textcolor{Mulberry}{TODO: Importance of regular category}

\subsection{Regular monomorphisms}
\textcolor{Mulberry}{TODO: Regular monos in logic + ref}

Recall that a morphism \(f \colon X \to Y\) is a \textbf{regular monomorphism}
if it fits in an equalizer diagram
\[
  \begin{tikzcd}
    X \ar[r,"f"]
    & Y \ar[r,shift left, "g"]\ar[r, shift right, "h"']
    & Z
  \end{tikzcd}
\]
for some morphisms \(g\) and \(h\).

\begin{exercise}[Characterization of regular monomorphisms]%
  \label{exer:characterize-regular-monos}
  Prove that an assembly map \(f \colon X \to Y\) is a regular monomorphism if
  and only if \(f \colon \carrier{X} \to \carrier{Y}\) is injective and there
  exists an element \(\pca{i} \in \AA\) such that
  \(\pca{i}\pca{b} \realizes_X x\) for all \(x \in \carrier{X}\) and
  \(\pca{b} \realizes_Y f(x)\).
\end{exercise}

\begin{exercise}\label{exer:mono-but-not-regular-mono}
  Give an example of a monomorphism in \(\Asm{\AA}\) which is not regular (for a
  nontrivial pca \(\AA\)).
\end{exercise}

In the category of sets, every function \(f \colon A \to B\) factors as a
regular epimorphism (= surjection) followed by a monomorphism (= injection):
\[
  \begin{tikzcd}[row sep=3mm,column sep=3mm]
    A \ar[dr,"\tilde f"'] \ar[rr,"f"] & & B \\
    & \im(f) \ar[ur,hookrightarrow]
  \end{tikzcd}
\]
with \(\im(f) \coloneq \set{b \in B \mid \exists (a \in A).f(a) = b}\).
%
The same is true in the category of assemblies, as we ask you to verify in the
following exercise.

\begin{exercise}\label{exer:reg-epi-mono-factorization}
  Given an assembly map \(f \colon X \to Y\), show how to factor it in
  \(\Asm{\AA}\) as a regular epimorphism followed by a monomorphism.
\end{exercise}

\section{Base change adjoints}

\begin{proposition}
  For every morphism \(f \colon X \to Y\) of assemblies, the pullback functor
  \(f^\ast \colon \Asm{\AA}/Y \to \Asm{\AA}/X\) has both a left adjoint
  \(\sum_f\) and a right adjoint \(\prod_f\).
\end{proposition}
\begin{proof}[Proof sketch]
  We only describe the constructions and leave the verification of the details
  to the interested reader.
  %
  The left adjoint \(\sum_f\) takes an object \(g \colon Z \to X\) of
  \(\Asm{\AA}/X\) to the object \(f \circ g\) of \(\Asm{\AA}/Y\).
  %
  On morphisms it is the identity.

  The right adjoint \(\prod_f\) is more involved.
  %
  Given an object \(g \colon Z \to X\) of \(\Asm{\AA}/X\), we consider the
  assembly \(P\) of ``fiberwise maps''. It is given by
  \[
    \carrier{P} \coloneq
    \set{(y,s) \mid s \colon f^{-1}(y) \to Z \text{ such that }
      \forall(x \in \carrier{f^{-1}(y)})\,.\,s(x) \in \carrier{g^{-1}(x)}},
  \]
  where
  \[
    \carrier{f^{-1}(y)} \coloneq \set{x \in \carrier{X} \mid f(x) = y}
    \quad\text{with realizers}\quad
    \pca{a} \realizes_{f^{-1}(y)} x \text{ if } \pca{a} \realizes_X x,
  \]
  (and similarly for \(g^{-1}(x)\)), and for realizers, we put
  \[
    \pcapair\pca{b}\pca{t} \realizes_P (y,s)
    \text{ if }
    \pca{b} \realizes_Y y
    \text{ and}
    \pca{t} \text{tracks } s.
  \]
  Now, \(P\) defines an object of \(\Asm{\AA}/Y\) by considering the first
  projection \(\pi_1 \colon P \to Y\) which is tracked by \(\pcafst\).
  %
  This projection map is the value of \(\prod_f(g)\).
  %
  Given a morphism
  \[
    \begin{tikzcd}[row sep=4mm,column sep=4mm]
      Z \ar[rr,"k"] \ar[dr,"g"'] & & W \ar[dl,"h"] \\
      & X
    \end{tikzcd}
  \]
  in \(\Asm{\AA}/X\), we define \(\prod_f(k) \colon \prod_f(g) \to \prod_f(h)\)
  as a function on sets by \((y,s) \mapsto (y, h \circ s)\). This assignment can
  be shown to be tracked because \(h\) and \(s\) are.
\end{proof}

\textcolor{Mulberry}{Mention and reference for Beck--Chevalley. Intuition:
  \(\prod\) and \(\sum\) ``respect'' substitution (which is
  pullback~\cite{Bauer2012}).}
% \begin{lemma}[Beck--Chevalley condition]
%   For every pullback square
%   \[
%     \begin{tikzcd}
%       X \times_Z Y \pbcorner \ar[r,"\pi_2"] \ar[d,"\pi_1"'] & Y \ar[d,"g"] \\
%       X \ar[r,"f"] & Y
%     \end{tikzcd}
%   \]
%   the natural transformations
%   \[
%     \sigma \colon \sum_{\pi_1}
%   \]
% \end{lemma}

\section{List of exercises}
\begin{enumerate}
\item \cref{exer:coproducts}: On the universal property of coproducts.
\item \cref{exer:coproduct-booleans}: On the assembly of booleans as the
  coproduct \(1 + 1\).
\item \cref{exer:nno}: On natural numbers objects.
\item \cref{exer:Gamma-global-sections}: On the forgetful functor \(\Gamma\) and
  the global sections functor.
\item \cref{exer:Gamma-global-sections}: On \(\Gamma\) being a left adjoint to
  \(\nabla\).
\item \cref{exer:no-nabla-to-Two}: On the nonexistence of assembly maps
  \(\nabla\set{0,1} \to \Two\).
\item \cref{exer:nabla-no-right-adjoint}: On the nonexistence of a right
  adjoint to \(\nabla\).
\item \cref{exer:Gamma-no-left-adjoint}: On the nonexistence of a left
  adjoint to \(\Gamma\).
\item \cref{exer:characterize-regular-epis}: On characterizing the regular
  epimorphisms.
\item \cref{exer:epi-but-not-regular-epi}: On an example of a epimorphism
  which is not a regular.
\item \cref{exer:characterize-regular-monos}: On characterizing the regular
  monomorphisms.
\item \cref{exer:mono-but-not-regular-mono}: On an example of a monomorphism
  which is not a regular.
\item \cref{exer:reg-epi-mono-factorization}: On factoring an assembly map as a
  regular epi followed by a mono.
\end{enumerate}

%%% Local Variables:
%%% mode: latexmk
%%% TeX-master: "../main"
%%% End:
