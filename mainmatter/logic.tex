\chapter{The realizability interpretation of logic}\label{chap:logic}

\section{Categorical logic in a nutshell}

\cite{Lawvere1969}

\section{Categorical logic in categories of assemblies}

\textcolor{Mulberry}{TODO: Intro}

\begin{definition}[Realizability predicate]
  A \textbf{realizability predicate} on an assembly~\(X\) is a function
  \(\carrier{X} \to \PA\).
\end{definition}

Note that the definition of a realizability predicate makes sense even when
\(X\) is just a set and not an assembly. However, we will shortly define a
perorder on realizability predicates on an assembly \(X\) which \emph{does} make
essential use of the realizers of \(X\).

\begin{notation}
  We will typically write \(\phi\) and \(\psi\) for realizability predicates.
\end{notation}

We think of a realizability predicate \(\phi\) on \(X\) as a logical predicate
on \(X\), and of \(\phi(x)\) as the set of witnesses that \(\phi\) holds for the
element \(x\).

\begin{definition}[Preorder on realizability predicates, \(\phi \predleq \psi\)]
  For two realizability predicates \(\phi\) and \(\psi\) on an assembly \(X\),
  we put
  \(
    \phi \predleq \psi
  \)
  exactly if there exists \(\pca{r} \in \AA\) such that,
  for every \(x \in \carrier{X}\), realizer \(\pca{a} \realizes_X x\) and witness \(\pca{b} \in \phi(x)\), we have
  \(\pca{r}\pca{a}\pca{b} \in \psi(x)\),
  %   \phi \predleq \psi \iff
  %   \exists(\pca{r} \in \pca{\AA}).
  %   \forall (x \in X).
  %   \forall(\pca{a} \realizes_X x).
  %   \forall(\pca{b} \in \phi(x)).
  %   \pca{r}\pca{a}\pca{b} \in \psi(x),
  % \]
  where we implicitly include the requirement that \(\pca{r}\pca{a}\pca{b}\) is defined.
\end{definition}

Intuitively, we have \(\phi \predleq \psi\) exactly if we can effectively
calculate a witness that \(\psi\) holds at \(x\) form a witness that \(\phi\)
holds at \(x\) \emph{and} a realizer of \(x\).
%
It is at this last point that we make essential use of the fact that \(X\) is an
assembly and not just a set.

%
It is straightforward (and very similar to checking that assemblies and assembly
maps form a category) to check that \(\predleq\) is indeed a preorder, i.e.\
that it is a reflexive and transitive relation.

Every monomorphism \(m \colon Y \hookrightarrow X\) on an assembly \(X\)
determines a realizability predicate \(\phi_m\) on \(X\) by:
\begin{align*}
  \phi_m(x) &\coloneqq
  \set{\pca{a} \in \AA \mid y \in m^{-1}(x) \text{ and} \pca{a} \realizes_Y y} \\
  &\hspace{3.5pt}=
  \set{\pca{a} \in \AA \mid
    \exists(y \in \carrier{Y}).\pca{a} \realizes_Y y \text{ and } m(y) = x}.
\end{align*}
%
Notice that the \(y \in \carrier{Y}\) is necessarily unique (if it exists),
because \(m\) is injective.

Conversely, every realizability predicate on \(X\) determines a monomorphism of
assemblies \([\phi] \hookrightarrow X\) via:
\begin{align*}
  \carrier{[\phi]} \coloneqq \set{x \in \carrier{X} \mid \phi(x) \text{ is inhabited}} \\
  \pcapair\pca{a}\pca{b} \realizes_{[\phi]} x
  \iff \pca{a} \realizes_X x
  \text{ and }
  \pca{b} \in \phi(x),
\end{align*}
where the inclusion \([\phi] \hookrightarrow X\) is tracked by \(\pcafst\).

\begin{proposition}\label{preorders-mono-predicates-iso}
  For any assembly \(X\), the above constructions constitute an isomorphism
  between the preorder of monomorphisms into \(X\) and the preorder of
  realizability predicates on \(X\).
\end{proposition}
\begin{exercise}\label{exer:preorders-monos-predicates-iso}
  Prove the proposition.
\end{exercise}


\subsection{The Heyting prealgebra of realizability predicates}

\begin{definition}
  There are two extreme examples of realizability predicates on an assembly \(X\):
  \begin{align*}
    \bot(x) &\coloneqq \emptyset, \\
    \top(x) &\coloneqq \AA.
  \end{align*}
  For realizability predicates \(\phi\) and \(\psi\) on a set \(X\), we define
  three new realizability predicates on \(X\) by:
  \begin{align*}
  %\item
    (\phi \land \psi)(x) &\coloneqq \set{\pcapair\pca{a}\pca{b} \mid
      \pca{a} \in \phi(x) \text{ and } \pca{b} \in \psi(x)}, \\
  %\item
    (\phi \lor \psi)(x) &\coloneqq
    \set{\pcaleft\pca{a} \mid \pca{a} \in \phi(x)} \cup
    \set{\pcaright\pca{b} \mid \pca{b} \in \psi(x)},
                          \quad(\text{recall~\cref{coproducts}}) \\
    % \item
    (\phi \Rightarrow \psi)(x) &\coloneqq
    \set{\pca{r} \in \AA \mid \text{for every}
    \pca{a} \in \phi(x) \text{ we have} \pca{r}\pca{a} \in \psi(x)}.
  \end{align*}
\end{definition}

\begin{exercise}\label{exer:predicates-heyting-prealgebra}
  Check that \(\bot\), \(\top\), \(\land\), \(\lor\) and \(\Rightarrow\) as
  defined above make the preorder of realizability predicates on an assembly
  \(X\) into a Heyting prealgebra.

  (For a short definition of the latter: it is a preorder that, when viewed as a
  category, has finite (co)limits and exponentials.)
\end{exercise}

In light of \cref{preorders-mono-predicates-iso} and the fact that the Heyting
(pre)algebra operations are characterized by universal properties we know that
the operations defined above correspond to the relevant operations on monos. For
example, \([\bot]\) is the monomorphism \(0 \hookrightarrow X\) and
\([\phi\land\psi]\) is the meet (= greatest lower bound) of the monos \([\phi]\)
and \([\psi]\).

\textcolor{Mulberry}{TODO: Quantifiers intro}

\begin{proposition}
  For an assembly map \(f \colon X \to Y\) and a realizability predicate
  \(\phi\) on \(X\), the realizability predicate \(\forall_f(\phi)\) on \(Y\)
  defined by
  \[
    {\forall_f(\phi)}\,(y) \coloneqq
    \set{\pca{t} \in \AA \mid
      \text{for every } x \in f^{-1}(y) \text{ and } \pca{a} \realizes_X x, \text{we have}
      \pca{t}\pca{a} \in \phi(x)}
  \]
  satisfies
  \[
    f^\ast(\psi) \predleq \phi \iff \psi \predleq \forall_f(\phi)
  \]
  for all realizability predicates \(\psi\) on \(Y\).

  In other words, \(\forall_f \colon \realpred(X) \to \realpred(Y)\) is a right
  adjoint to \(f^\ast \colon \realpred(Y) \to \realpred(X)\).
\end{proposition}
\begin{proof}
  We spell out what each of \(f^\ast(\psi) \predleq \phi\) and
  \(\psi \predleq \forall_f(\phi)\) amounts to.

  The former requires the existence of an element \(\pca{r_1} \in \AA\) such that
  for every \(x \in \carrier{X}\), \(\pca{a} \realizes_X x\) and
  \(\pca{c} \in \psi(f(x))\), we have \(\pca{r_1}\pca{a}\pca{c} \in \phi(x)\).

  The latter requires the existence of an element \(\pca{r_2} \in \AA\) such
  that for every \(y \in \carrier{Y}\), \(\pca{b} \realizes_Y y\) and
  \(\pca{c} \in \psi(y)\) we have
  \(\pca{r_2}\pca{b}\pca{c} \in \forall_f(\phi)\,(y)\).
  %
  That is, \(\pca{r_2}\pca{b}\pca{c}\) should satisfy
  \(\pca{r_2}\pca{b}\pca{c}\pca{a} \in \phi(x)\) for all
  \(\pca{a} \realizes_X x\) with \(f(x) = y\).

  Now notice that given such an \(\pca{r_1}\), the program
  \(\lambdapca{vwu}{\pca{r_1}u\,w}\) does the job of \(\pca{r_2}\).

  Conversely, given such an \(\pca{r_2}\), the program
  \(\lambdapca{uw}{\pca{r_2}(\pca{t_f}u)\,w\,u}\), where \(\pca{t_f}\) is a
  tracker of \(f\), does the job of \(\pca{r_1}\).
\end{proof}

For an alternative proof, one may verify that \(\forall_f(\phi)\) is the
realizability predicate determined by the monomorphism \(\forall_f([\phi])\).

\begin{proposition}
  For an assembly map \(f \colon X \to Y\) and a realizability predicate
  \(\phi\) on \(X\), the realizability predicate \(\exists_f(\phi)\) on \(Y\)
  defined by
  \[
    {\exists_f(\phi)}\,(y) \coloneqq
    \bigcup_{x \in f^{-1}(y)}\set{\pcapair\pca{a}\pca{b}
      \mid \pca{a} \realizes_X x \text{ and } \pca{b} \in \phi(x)}
  \]
  satisfies
  \[
    \exists_f(\phi) \predleq \psi \iff \phi \predleq f^\ast(\psi)
  \]
  for all realizability predicates \(\psi\) on \(Y\).

  In other words, \(\exists_f \colon \realpred(X) \to \realpred(Y)\) is a left
  adjoint to \(f^\ast \colon \realpred(Y) \to \realpred(X)\).
\end{proposition}
\begin{exercise}\label{exer:exists-predicate}
  Prove the proposition either directly or by checking that \(\exists_f(\phi)\)
  is the realizability predicate \(\phi_m\) determined by \(m\) in the
  factorization of the top composite
  \[
    \begin{tikzcd}[row sep=2mm,column sep=8mm]
      {[\phi]} \ar[dr] \ar[r,hookrightarrow] & X \ar[r,"f"] & Y \\
      & M \ar[ur,hookrightarrow,"m"']
    \end{tikzcd}
  \]
  as a regular epimorphism followed by a monomorphism
  (recall~\cref{exer:reg-epi-mono-factorization}).
  %
  In other words, that \(\exists_f(\phi)\) is the realizability predicate
  corresponding to the mono \(\exists_f([\phi])\).
\end{exercise}

\section{Revisiting (regular) monos and epis}

\begin{exercise}\label{exer:epis-monos-logically}
  Prove the following logical characterizations for any assembly map \(f \colon X \to Y\):
  \begin{enumerate}[(i)]
  \item \(f\) is a regular epimorphism if and only if
    \[
      \forall(x : X).\exists(y : Y).f(x) = y
    \]
    is realized;
  \item \(f\) is an epimorphism if and only if
    \[
      \forall(x : X).\lnot\lnot({\exists(y : Y).f(x) = y})
    \]
    is realized;
  \item \(f\) is a monomorphism if and only if
    \[
      \forall(x, x' : X).(f(x) = f(x') \Rightarrow x = x')
    \]
    is realized;
  \item \(f\) is a regular monomorphism if and only if
    \[
      \forall(y : Y).\pa*{\lnot\lnot\pa*{\exists(x : X).f(x) = y} \Rightarrow \exists!(x : X).f(x) = y}
    \]
    is realized.

    The quantifier \(\exists!\) means ``there exists a unique \dots with \dots''.

    Phrased in English, \(f\) is a regular monomorphism if and only if the statement
    \begin{quote}{``For all \(y\), if the primage of \(f\) at \(y\) is nonempty,
        then we can (effectively) find a unique \(x\) with \(f(x) = y\).''}
    \end{quote}
    is realized.
  \end{enumerate}
\end{exercise}

The \(\lnot\lnot\)-stable predicates have no computational content as made
precise by the following exercise:

\begin{exercise}\label{exer:not-not-stable-nabla-pullback}
  Show that every \(\lnot\lnot\)-stable predicate \(\phi\) on an assembly \(X\)
  is uniquely determined by a subset \(A \subseteq \carrier{X}\) such that we
  have a pullback diagram
  \[
    \begin{tikzcd}
      {[\phi]} \ar[r] \ar[d,hookrightarrow] \pbcorner
      & \nabla A \ar[d,hookrightarrow] \\
      X \ar[r,"\eta_X"] & \nabla\carrier{X}
    \end{tikzcd}
  \]
  in \(\Asm{\AA}\), where we recall that \([\phi] \hookrightarrow X\) denotes
  the mono corresponding to \(\phi\).
\end{exercise}

In fact, because subsets are uniquely determined (again, via pullback) by their
characteristic function, we have:

\begin{exercise}\label{exer:not-not-stable-nabla-two-pullback}
  Show that every \(\lnot\lnot\)-stable predicate \(\phi\) on an assembly \(X\)
  is uniquely determined by a map \(\chi \colon X \to \nabla\set{0,1}\) such
  that we have a pullback diagram
  \[
    \begin{tikzcd}
      {[\phi]} \ar[r] \ar[d,hookrightarrow] \pbcorner
      & 1 \ar[d,hookrightarrow,"{\singleton \,\mapsto\, 1}"] \\
      X \ar[r,"\chi"] & \nabla\set{0,1}
    \end{tikzcd}
  \]
  in \(\Asm{\AA}\).
\end{exercise}

The map \(1 \to \nabla\set{0,1}\) is said to be a \textbf{classifier} for the
\(\lnot\lnot\)-stable predicates (subobjects).

\section{List of exercises}
\begin{enumerate}
\item \cref{exer:preorders-monos-predicates-iso}: On monomorphisms as
  realizability predicates and vice versa.
\item \cref{exer:predicates-heyting-prealgebra}: On the Heyting prealgebra of
  realizability predicates.
\item \cref{exer:exists-predicate}: On \(\exists_f\) being left adjoint to
  \(f^\ast\) as maps between preorders of realizability predicates.
\item \cref{exer:epis-monos-logically}: On logical characterizations of
  (regular) epimorphisms and monomorphisms.
\item \cref{exer:not-not-stable-nabla-two-pullback}: On the correspondence
  between \(\lnot\lnot\)-stable predicates and subsets.
\item \cref{exer:not-not-stable-nabla-two-pullback}: On the classifier for
  \(\lnot\lnot\)-stable predicates.
\end{enumerate}

%%% Local Variables:
%%% mode: latexmk
%%% TeX-master: "../main"
%%% End:
