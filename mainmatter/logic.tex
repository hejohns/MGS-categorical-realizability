\chapter{The realizability interpretation of logic}\label{chap:logic}

\begin{exercise}\label{exer:epis-monos-logically}
  Prove the following logical characterizations for any assembly map \(f \colon X \to Y\):
  \begin{enumerate}[(i)]
  \item \(f\) is a regular epimorphism if and only if
    \[
      \forall(x : X).\exists(y : Y).f(x) = y
    \]
    is realized;
  \item \(f\) is an epimorphism if and only if
    \[
      \forall(x : X).\lnot\lnot({\exists(y : Y).f(x) = y})
    \]
    is realized;
  \item \(f\) is a monomorphism if and only if
    \[
      \forall(x, x' : X).(f(x) = f(x') \Rightarrow x = x')
    \]
    is realized;
  \item \(f\) is a regular monomorphism if and only if
    \[
      \forall(y : Y).\pa*{\lnot\lnot\pa*{\exists(x : X).f(x) = y} \Rightarrow \exists!(x : X).f(x) = y}
    \]
    is realized.

    The quantifier \(\exists!\) means ``there exists a unique \dots with \dots''.

    Phrased in English, \(f\) is a regular monomorphism if and only if the statement
    \begin{quote}{``For all \(y\), if the primage of \(f\) at \(y\) is nonempty, then we can
      (effectively) find a unique \(x\) with \(f(x) = y\).''}\end{quote}
    is realized.
  \end{enumerate}
\end{exercise}

\section{List of exercises}
\begin{enumerate}
\item \cref{exer:epis-monos-logically}: On logical characterizations of
(regular) epimorphisms and monomorphisms.
\end{enumerate}

%%% Local Variables:
%%% mode: latexmk
%%% TeX-master: "../main"
%%% End:
