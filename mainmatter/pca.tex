\chapter[Models of computation: partial combinatory algebras]{Models of computation: \\ partial combinatory algebras}\label{chap:PCA}

The starting ingredient in categorical realizability is a general model of
computation known as a \emph{partial combinatory algebra}, or \emph{pca} for
short.
%
The desire for a general model of computation is motivated by the many possible
interesting examples
(see~\cref{sec:basic-examples-of-pcas,sec:more-examples-of-pcas}).
%
Moreover, in the next chapter, we construct the \emph{category of assemblies}
over a fixed but arbitrary pca. The construction itself works generally and is
insensitive to any particular choice of pca. However, the choice of pca is
reflected in the logical principles that the resulting category of assemblies
validates.
%
Thinking of categories of assemblies as ``worlds of computable mathematics'', we
can thus build different worlds by varying our notion of pca.

Having said all this, in these notes we have chosen to work with the
\emph{untyped} notion of a pca. The \emph{typed} notion, as developed by
Longley\footnote{Fun fact: John Longley is also the author of the recent
  \emph{Castles in the Air}---an introduction to the world of mathematical logic
  cast in the form of a fantasy novel---as well as a semi-professional
  pianist.} in his PhD thesis~\cite{Longley1995}, is more general and moreover
has the advantage that it can capture (idealized) typed functional programming
languages (such as PCF~\cite{Plotkin1977,deJong2023}).
%
We only treat untyped pcas for simplicity and refer the interested reader to
Bauer's more comprehensive notes~\cite{Bauer2023}.


\begin{definition}[Partial combinatory algebra (pca), \(\kcomb\), \(\scomb\)]\label{def:pca}
  A \textbf{partial combinatory algebra} (\textbf{pca}) is a set \(\AA\) together with a \emph{partial}
  operation \(\AA \times \AA \pto \AA\), denoted by juxtaposition,
  \((\pca{a},\pca{b}) \mapsto \pca{a}\pca{b}\), such that there exist elements \(\kcomb\) and \(\scomb\)
  satisfying:
  \begin{enumerate}[(i)]
  \item\label{k-behaviour} \((\kcomb \pca{a}) \pca{b} = \pca{a}\) for all elements
    \(\pca{a},\pca{b} \in \AA\),
  \item\label{s-defined} \(((\scomb \pca{f}) \pca{g})\) is defined for all elements
    \(\pca{f},\pca{g} \in \AA\), and
  \item\label{s-behaviour}
    \(((\scomb \pca{f}) \pca{g}) \pca{a} \simeq
    (\pca{f}\pca{a})(\pca{g}\pca{a})\) for all elements
    \(\pca{f},\pca{g},\pca{a} \in \AA\).
  \end{enumerate}
  The symbol \(\simeq\) is \emph{Kleene equality} and means: either both sides
  are undefined, or both are defined and are equal elements of \(\AA\).
  %
  In particular, in~\ref{k-behaviour}, the operation \(\kcomb\pca{a}\) must be
  defined for all elements \(\pca{a} \in \AA\).
\end{definition}

We think of the elements of a pca as codes for programs that can also act as
input. Accordingly, we pronounce \(\pca{a}\pca{b}\) as ``\(\pca{a}\) applied to
\(\pca{b}\)'' and think of this as: apply the program with code \(\pca{a}\) to
input \(\pca{b}\).

In this light, the element \(\kcomb\), which we call the \emph{k-combinator}, act
as a parameterized constant program: it takes an input \(\pca{a}\) and then
always outputs \(\pca{a}\) on any input \(\pca{b}\).

The element \(\scomb\), which we call the \emph{s-combinator}\footnote{The
  letters \emph{k} and \emph{s} come from Moses Sch\"onfinkel's combinatory
  logic~\cite{Schonfinkel1924}. They respectively come from the German words
  \emph{Konstanzfunktion} (constant function) and \emph{Verschmelzungfunktion}
  (merge function). Of course, \emph{Verschmelzungsfunktion} starts with a
  \emph{v}, but Sch\"onfinkel had to avoid confusion as there was also a
  swap-arguments combinator called the \emph{Vertauschungsfunktion}.}, act as
parameterized application: it takes two codes for programs \(\pca{f}\) and
\(\pca{g}\) and an input \(\pca{a}\) and then applies \(\pca{f}\pca{a}\) to
\(\pca{g}\pca{a}\).

\begin{notation}
  We will economize on parentheses and write \(\pca{a}\pca{b}\pca{c}\) for
  \((\pca{a}\pca{b})\pca{c}\).
  %
  We will always use the \texttt{teletype font} for arbitrary elements of a pca
  to reinforce the idea that these should be thought of as (codes of) programs.
  %
  For fixed programming constructs, we will use this \(\pcacomb{bold\; font}\)
  instead.
\end{notation}

The point of the k- and s-combinators is that they provide us with a (very
minimal) programming interface which we will explore
in~\cref{sec:basic-programming-in-pcas}.
%
We prefer to give two examples first to strengthen our intuition for pcas.

\section{Basic examples of pcas}\label{sec:basic-examples-of-pcas}

To help build some intuition for pcas, we now consider three basic examples. The
first example is a triviality, but we include it here because the category of
assemblies (see~\cref{chap:assemblies}) over it is equivalent to the familiar category of
sets.

\begin{example}[The trivial pca]
  The \textbf{trivial pca} is the singleton set \(\set{\singleton}\) with
  application map \((\singleton,\singleton) \mapsto \singleton\).
  %
  Of course, \(\kcomb \coloneq \singleton\) and \(\scomb \coloneq \singleton\).
\end{example}

\begin{example}[Untyped \(\lambda\)-calculus as a pca, \(\Lambda\)]
  Write \(\Lambda\) for the closed terms of the untyped \(\lambda\)-calculus
  quotiented by the equivalence relation generated by
  \(\beta\)-reduction. (E.g., for a closed \(\lambda\)-term \(t\), we identify
  \((\lambda{x}.{x})t\) and \(t\).)
  %
  With \(\lambda\)-calculus application the set \(\Lambda\) forms a pca with
  \(\kcomb\) and \(\scomb\) given by the equivalence classes of
  \(\lambda{xy}.x\) and \(\lambda{xyz}.(xz)(yz)\), respectively.
\end{example}

The application function of \(\Lambda\) is actually total, i.e.\ it is defined
on any two inputs. An example of a pca with a genuine partial application---which
is also the prime example of a pca---is \emph{Kleene's first model}~\cite{Kleene1945}:

\begin{example}[Kleene's first model, \(\Kone\)]\label{ex:Kleene-1}
  We define a partial application function on the set of natural numbers
  \(\Nat\): for natural numbers \(n\) and \(m\) we take \(n\,m\) to be
  \(\prenum{n}(m)\), where \(\prenum{-}\) denotes a Turing computable
  enumeration of the Turing computable partial functions on the natural numbers.

  The existence of the k- and s-combinators follows from Kleene's
  \(S^m_n\)-theorem in computability theory, see e.g.~\cite[Section~2.5.1 and
  Theorem~2.1.5]{Bauer2023} for details.

  We write \(\Kone\) for this partial combinatory algebra and return to
  it in~\cref{chap:logic,chap:synthetic}.
\end{example}

\section{Basic programming in pcas}\label{sec:basic-programming-in-pcas}

%As previously mentioned,
The k- and s-combinators of a pca provide an
interface for writing basic programs. For example, if we put\footnote{Note
  that this indeed defines an element of \(\AA\)
  by~\cref{def:pca}\ref{s-defined}.}
\(\icomb \coloneq \scomb\kcomb\kcomb\) then \(\icomb\) acts as the
identity combinator:
\(
  \icomb\pca{a} = \scomb\kcomb\kcomb\pca{a} = \kcomb\pca{a}(\kcomb\pca{a}) =
  {\pca{a}}
  \)
  for any element \(\pca{a}\) of our pca.

While theoretically possible, it is very inconvenient to program everything
directly in terms of \(\kcomb\) and \(\scomb\). Therefore, we will
define something resembling \(\lambda\)-abstraction for our pca, allowing us to
write the clearer \(\icomb \coloneq \lambdapca{x}{\var{x}}\) instead.

\begin{definition}[Terms]% over a pca]
  We fix a countably infinite set of variables, typically denoted by
  \(x,y,z,u,v,w,x_0,x_1,\dots\), and inductively define the set of
  \textbf{terms} over a pca~\(\AA\):
  \begin{enumerate}[(i)]
  \item a variable is a term,
  \item an element of \(\AA\) is a term,
  \item given two terms \(s\) and \(t\), we may form a new term denoted by \(s\,t\).
  \end{enumerate}
  A term without variables is called \textbf{closed}.
\end{definition}

\begin{notation}
  If \(t\) is a term, then we write \(t[\pca{a_1}/x_1,\dots,\pca{a_n}/x_n]\) for
  the \emph{closed} term obtained by substituting the element
  \(\pca{a_i} \in \AA\) for the variable \(x_i\) (which may or may not occur in
  \(t\)).
\end{notation}

\begin{definition}[Defined terms]
  A closed term is \textbf{defined} if, when we interpret \({s\,t}\) as
  \(s\) applied to \(t\) in \(\AA\), all these applications are defined.

  We extend this to terms with variables: such a term \(t\) is \textbf{defined}
  if for all possible substitutions of all variables in \(t\) by elements of
  \(\AA\), the resulting closed term obtained via substitution is defined.
\end{definition}

For example, the terms \(\kcomb\scomb\icomb\) and \(\scomb x\,y\) are defined.

\begin{notation}
  We extend Kleene equality from closed terms to all terms over \(\AA\): given
  two terms \(s\) and \(t\) whose variables are among \(x_1,\dots,x_n\), we
  write \(t_1 \simeq t_2\) if for all \(\pca{a_1},\dots,\pca{a_n} \in \AA\), we
  have
  \[
    t_1[\pca{a_1}/x_1,\dots,\pca{a_n}/x_n] \simeq
    t_2[\pca{a_1}/x_1,\dots,\pca{a_n}/x_n]
  \]
  as closed terms.
\end{notation}

We now define something resembling \(\lambda\)-abstraction for pcas.

\begin{definition}[``\(\lambda\)-abstraction'' in a pca, \(\lambdapca{x}{t}\)]
  For a variable \(x\) and a term \(t\), we define a new term, denoted by
  \(\lambdapca{x}{t}\), by recursion on terms:
  \begin{itemize}
  \item \(\lambdapca{x}{x} \coloneqq \icomb = \scomb\kcomb\kcomb\),
  \item \(\lambdapca{x}{y} \coloneqq \kcomb y\) if \(y\) is a variable different from \(x\),
  \item \(\lambdapca{x}{\pca{a}} \coloneqq \kcomb\pca{a}\) for \(\pca{a} \in \AA\),
  \item \(\lambdapca{x}{(t_1\,t_2)} \coloneqq \scomb(\lambdapca{x}{t_1})\,(\lambdapca{x}{t_2})\).
  \end{itemize}
\end{definition}

\begin{exercise}[Combinatory completeness]\label{exer:properties-of-lambda-pca}
  Prove that for every variable \(x\) and term~\(t\), the following properties
  of \(\lambdapca{x}{t}\) hold:
  \begin{enumerate}[(i)]
  \item The variables of \(\lambdapca{x}{t}\) are exactly those of \(t\) minus
    \(x\).
  \item\label{lambda-pca-def} The term \(\lambdapca{x}{t}\) is defined.
  \item For all \(\pca{a} \in \AA\), we have
    \((\lambdapca{x}{t})\pca{a} \simeq t[\pca{a}/x]\).
  \end{enumerate}
\end{exercise}

\begin{notation}
  We write \(\lambdapca{xy}{t}\) for the term
  \(\lambdapca{x}({\lambdapca{y}{t}})\) and similarly for more variables.
\end{notation}

Suppose we have a term \(t\) featuring only the variables \(x\), \(y\) and
\(z\), e.g., \(t \coloneqq z\kcomb x(\kcomb y)\).
%
We think of \(t\) as a \emph{partial} function
\(\AA \times \AA \times \AA \pto \AA\), taking three inputs \(\pca{a}\),
\(\pca{b}\) and \(\pca{c}\) that we may substitute in \(t\) for \(x\), \(y\) and
\(z\), respectively, and that results in the term
\(\pca{c}\kcomb\pca{a}(\kcomb\pca{b})\) which has no variables.
%
Combinatory completeness says that terms can be internalized in a pca. Indeed,
we can represent \(t\) in \(\AA\) as \(\lambdapca{xyz}{t}\). Note that this is
indeed an element of \(\AA\) by virtue
of~\cref{exer:properties-of-lambda-pca}\ref{lambda-pca-def} and the fact that
\(\lambdapca{xyz}{t}\) is a closed term.

% \begin{theorem}[Combinatory completeness]
%   For every term \(t\) with variables \(x_1,\dots,x_{n+1}\), there is an element
%   \(\pca{a} \in \AA\) such that for all
%   \(\pca{b_1},\dots,\pca{b_{n+1}} \in \AA\), we have:
%   \begin{enumerate}[(i)]
%   \item \(\pca{a}\pca{b_1}\cdots\pca{b_n}\) is defined, and
%   \item \(\pca{a}\pca{b_1}\cdots\pca{b_n}\pca{b_{n+1}}\)
%   \end{enumerate}
% \end{theorem}
% \begin{proof}
% \end{proof}

\begin{remark}
  While similar to \(\lambda\)-abstraction, we deliberately do not use the
  \(\lambda\)-symbol, because it might suggest that \(\lambdapca{x}{t}\) obeys the
  \emph{\(\beta\)-law} when substituting terms, but due to the partial nature of
  the application map, this need not be the case~\cite[p.~4]{vanOosten2008}.
\end{remark}

It is now time to make use of combinatory completeness to construct some more
combinators. We will do this in very much the same way as one encodes these
constructs in the untyped \(\lambda\)-calculus.

\subsubsection*{Booleans and conditional}\label{sec:booleans}

We define the booleans as the closed terms
\[
  \pcatrue \coloneqq \lambdapca{xy}{x} \quad\text{and}\quad \pcafalse \coloneqq
  \lambdapca{xy}{y},
\]
and the conditional as the closed term
\[
  \pcaif \coloneq \lambdapca{x}{x}.
\]
%
Note that for arbitrary \(\pca{a}\), \(\pca{b} \in \AA\), we can calculate:
\[
  \pcaif\pcatrue\pca{a}\pca{b} = \pcatrue\pca{a}\pca{b} = \pca{a}
  \quad
  \text{and}
  \quad
  \pcaif\pcafalse\pca{a}\pca{b} = \pcafalse\pca{a}\pca{b} = \pca{b}.
\]

\subsubsection*{Pairing and projection}

We define the closed terms
\[
  \pcapair \coloneqq \lambdapca{xyz}{z}{xy},
  \quad
  \pcafst \coloneqq \lambdapca{w}{w\pcatrue}
  \quad
  \text{and}
  \quad
  \pcafst \coloneqq \lambdapca{w}{w\pcafalse}.
\]

\begin{exercise}\label{exer:pairing-projection}
  For all elements \(\pca{a}\), \(\pca{b} \in \AA\), prove that:
  \begin{enumerate}[(i)]
  \item \(\pcapair \pca{a} \pca{b}\) is defined.
  \item \(\pcafst{(\pcapair\pca{a}\pca{b})} = \pca{a}\) and
    \(\pcasnd{(\pcapair\pca{a}\pca{b})} = \pca{b}\) hold.
  \end{enumerate}
\end{exercise}

\subsubsection*{Fixed points}

We also have the fixed point combinators which are traditionally denoted by
\(\pcay\) and \(\pcaz\):
\begin{align*}
  \pcay &\coloneqq \pcaw\pcaw &\text{with}&& \pcaw &\coloneq \lambdapca{xy}{y\,(x\,x\,y)} \\
  \pcaz &\coloneqq \pcau\pcau &\text{with}&& \pcau &\coloneq \lambdapca{xyz}{y\,(x\,x\,y)\,z}
\end{align*}
These combinators satisfy:
\[
  \pcay\pca{f} \simeq \pca{f}{(\pcay\pca{f})},
    \quad
  \pcaz\pca{f} \text { is defined}
    \quad\text{and}\quad
  \pcaz\pca{f}\pca{a} \simeq \pca{f}{(\pcaz\pca{f})}\pca{a}
\]
for all elements \(\pca{f},\pca{a} \in \AA\).

\subsubsection*{Curry numerals and fundamental arithmetic}\label{sec:numerals}

For each natural number \(n \in \Nat\), we define the corresponding
\textbf{Curry numeral} \(\numeral{n}\) inductively by:
\[
  \numeral{0} \coloneq \icomb
  \quad\text{and}\quad
  \numeral{n+1} \coloneq \pcapair\pcafalse\numeral{n}.
\]

\begin{exercise}\label{exer:arithmetic}
  Define closed terms \(\pcasucc\), \(\pcapred\) and \(\pcaiszero\) such that
  for any natural number \(n \in \Nat\) the following equations hold:
  \begin{align*}
    \pcasucc\numeral{n} = \numeral{n+1},
    \quad
    \pcapred\numeral{0} = \numeral{0},
    \quad
    \pcapred\numeral{n+1} = \numeral{n}, \\
    %\quad
    \pcaiszero\numeral{0} = \pcatrue
    \quad\text{and}\quad
    \pcaiszero\numeral{n+1} = \pcafalse.
  \end{align*}

\end{exercise}

\subsubsection*{Primitive recursion}

The following \textbf{primitive recursion} combinator for Curry numerals will
come in useful when we consider the natural numbers object in the category of
assemblies later (\cref{TODO}).

\begin{exercise}\label{exer:primitive-recursion}
  Construct a closed term \(\pcarec\) such that for all
  \(\pca{f},\pca{a} \in \AA\) and \(n \in \Nat\) it satisfies:
  \[
    \pcarec\pca{a}\pca{f}\numeral{0} = a
    \quad\text{and}\quad
    \pcarec\pca{a}\pca{f}\numeral{n+1} \simeq \pca{f}\numeral{n}{(\pcarec\pca{a}\pca{f}\numeral{n})}
  \]
  \emph{Hint:} The essential ingredients are a zero test, predecessor and
  repeated application which are provided by \(\pcaiszero\), \(\pcapred\) and
  \(\pcaz\), respectively.
\end{exercise}

As a final remark, we note that any nontrivial pca is necessarily infinite, as
the following exercise shows:
\begin{exercise}\label{exer:nontrivial-pca}
  Prove that the following are equivalent for any pca \(\AA\):
  \begin{enumerate}[(i)]
  \item The booleans \(\pcatrue\) and \(\pcafalse\) are distinct elements of
    \(\AA\).
  \item The Curry numerals \(\numeral{n}\) in \(\AA\) are all distinct.
    % , i.e., the map
    % \(n \in \Nat \mapsto \numeral{n} \in \AA\) is an injection.
  \item The pca \(\AA\) is not the trivial pca.
  \end{enumerate}
\end{exercise}

\section{More examples of pcas}\label{sec:more-examples-of-pcas}

In this section we will present a few more examples of partial combinatory
algebras. To motivate and introduce them, we offer the following informal
explanation. Suppose we have a device which transforms the pitch of incoming
audio. We might represent the incoming and outgoing audio as streams of bits, so
that mathematically speaking, our in- and output are elements of
\(\set{0,1}^\Nat\), i.e.\ functions from the set of natural numbers \(\Nat\) to
the two-element set \(\set{0,1}\).
%
Our bit-stream transforming device is then a function
\(\set{0,1}^\Nat \to \set{0,1}^\Nat\).
%
Since we don't want to wait forever on the output of our device, it seems
reasonable that we assume it to start outputting after having only received
finitely many bits.
%
Thus, \emph{the output depends on a finite amount of the input only}.
%
Our device should therefore not be any old function
\(\set{0,1}^\Nat \to \set{0,1}^\Nat\), but rather one whose output depends on a
finite amount of input only.
%
This can be made mathematically precise by equipping the set \(\set{0,1}^\Nat\)
with a \emph{topology} and by restricting our attention to \emph{continuous}
functions on such topologized sets.

For an introduction to and motivation of topology from a computer science
perspective, Smyth's~\cite{Smyth1992} and Escard\'o's~\cite{Escardo2004}
monographs, and Vickers's book~\cite{Vickers1996} are warmly recommended.

While Turing computable functions, or equivalently, Kleene's partial recursive
functions, provide the foundation for computability theory with (encodings of)
finite objects, an established theory of computability over infinite objects is
Weihrauch's \emph{Type Two Effectivity (TTE)}~\cite{Weihrauch2000}.
%
This is a theory of computable analysis and becomes especially relevant when we
are interested in exact real number computation.
%
Its connections to realizability were explored in the PhD theses of
Lietz~\cite{Lietz2004} and Bauer~\cite{Bauer2000} via Kleene's second model and
its recursive variant (\cref{Kleene-2,Kleene-rec-2} below).

We will be brief in our explanations of these partial combinatory algebras and hope
that the interested reader will take the above paragraphs and the examples as an
invitation to explore the fascinating connections between topology and
computability, for instance by consulting Bauer's aforementioned notes on
realizability~\cite{Bauer2023} and the references above.

Our first example is due to Scott~\cite{Scott1976} and gives the powerset of the
natural number the structure of a pca.

\begin{example}[Scott's graph model, \(\Scott\)]
  We make the powerset \(\PN\) of the natural numbers into a pca that we denote
  by \(\Scott\).

  The idea behind application on \(\PN\) is that when we apply a subset \(U\) to
  a subset \(V\), then \(U\) can only ``use'' a finite amount of \(V\) to
  determine whether to include a number or not. We now make this idea formal.

  We first fix bijections
  \[
    \pairing{{-},{-}} \colon \Nat \times \Nat \to \Nat
    \quad\quad\text{and}\quad\quad
    \enum \colon \Nat \cong \powerset_{\operatorname{fin}}(\Nat)
  \]
  where the latter enumerates all finite subsets of natural numbers.
  %
  We then define the application map \(\PN \times \PN \to \PN\) as
  \[
    (U,V) \mapsto \set{\outputnum \in \Nat \mid
            \exists(\inputnum \in \Nat) .
            \pairing{\inputnum,\outputnum} \in U \text{ and } \enum(\inputnum) \subseteq V}.
  \]
  Thus, the ``program'' \(U\) encodes a list of input-output pairs where the
  input number encodes a finite subset of the argument \(V\).

  If we equip \(\PN\) with the \emph{Scott topology} whose basic opens are
  finite subsets of \(\Nat\), then one can show that the application map is
  continuous.
  %
  In fact, any continuous function \(F : \PN^k \to \PN\) can be represented in
  \(\PN\) by encoding the graph of \(F\).
  %
  (The details are spelled out in~\cite[Example~2.3.4]{deJong2018}.)
  %
  Under this correspondence it becomes easy to construct the elements \(\kcomb\)
  and \(\scomb\) making \(\PN\) into a pca.
\end{example}

The application in the above example is actually a \emph{total} operation. An
analogous pca whose application is genuinely partial is due to
Kleene~\cite{KleeneVesley1965}:
\begin{example}[Kleene's second model, \(\Ktwo\)]\label{Kleene-2}
  Somewhat similar to the above example, we can define an application on the set
  \(\Nat^\Nat\) of functions on \(\Nat\) that makes it into a pca~\(\Ktwo\),
  known as Kleene's second model.
  %
  The encodings required for the application map are a bit involved, see
  e.g.~\cite[p.~30 and Section~2.1.2]{Bauer2023} or
  \cite[Section~1.4.3]{vanOosten2008}, but we point out that this pca is closely
  related to continuous functions \(\Nat^\Nat \to \Nat^\Nat\) where we equip
  \(\Nat^\Nat\) with the \emph{Baire topology}.
\end{example}

Another example comes from \emph{domain theory}~\cite{AmadioCurien1998} and is
also due to Scott~\cite{Scott1972}:
\begin{example}[Scott's domain model of the untyped \(\lambda\)-calculus]
  The carrier of this pca is Scott's domain \(D_\infty\) which is a model of the
  untyped \(\lambda\)-calculus via the isomorphism
  \(\Phi \colon D_\infty \simeq {D_\infty}^{D_\infty}\) of domains.
  %
  The map~\(\Phi\) sends an element \(\sigma \in D_\infty\) to a Scott
  continuous function \(\Phi(\sigma) \colon D_\infty \to D_\infty\).
  %
  We define application on \(D_\infty\) as:
  \[
    (\sigma,\tau) \mapsto \Phi(\sigma)(\tau).
  \]
\end{example}

Finally, we mention that Scott's graph model and Kleene's second model have
effective variations that are examples of \emph{elementary sub-pcas}.

\begin{definition}[Elementary sub-pca]
  An \textbf{elementary sub-pca} of a pca \(\AA\) is a subset
  \(\AA^\# \subseteq \AA\) that is closed under the application of \(\AA\) and
  moreover contains the elements \(\kcomb\) and \(\scomb\) from \(\AA\).
\end{definition}

\begin{example}[Scott's r.e. graph model, \(\Scottre\)]
  We get an elementary sub-pca \(\Scottre\) of~\(\Scott\) by restricting
  ourselves to the recursively enumerable subsets of \(\Nat\).
\end{example}

\begin{example}[Kleene's recursive second model, \(\Ktworec\)]\label{Kleene-rec-2}
  We get an elementary sub-pca \(\Ktworec\) of \(\Ktwo\) by restricting
  ourselves to the total recursive %(= Turing computable)
  functions
  \(\Nat \to \Nat\).
\end{example}

For even more examples of pcas,
see~\cite[Section~1.4]{vanOosten2008}.

\section{List of exercises}
\begin{enumerate}
\item \cref{exer:properties-of-lambda-pca}: On the fundamental properties of
  \(\lambdapca{x}{t}\).
\item \cref{exer:properties-of-lambda-pca}: On the properties of the pairing and
  projection combinators.
\item \cref{exer:arithmetic}: On fundamental arithmetic in a pca.
\item \cref{exer:primitive-recursion}: On primitive recursion in a pca.
\item \cref{exer:nontrivial-pca}: On nontriviality of a pca.
\end{enumerate}




%%% Local Variables:
%%% mode: latexmk
%%% TeX-master: "../main"
%%% End:
