\chapter{Abstract}

Realizability, as invented by Kleene, is a technique for elucidating the
computational content of mathematical proofs. In this course we study
realizability from a categorical perspective.
%
Starting from an abstract and general model of computation known as a partial
combinatory algebra (pca), we construct the category of assemblies
over it.
%
Intuitively, an assembly is a set together with computability data and an
assembly map is a function of sets that is computable. Here, the notion of
computability is prescribed by the pca.
%
Through the framework of categorical logic, the assemblies give rise to the
realizability interpretation of logic, which we spell out in detail.

The central theme of this course is the interplay between category theory, logic
and computability theory. While some familiarity with basic category theory
(e.g.\ (co)limits and adjunctions) is required for some parts of the notes, the
course hopefully offers plenty to those unfamiliar with category theory.


%%% Local Variables:
%%% mode: latexmk
%%% TeX-master: "../main"
%%% End:
